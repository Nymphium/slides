\documentclass[unicode,compress,14pt,CJK%
\directlua{
    handout = os.getenv"HANDOUT"
    local _ = handout and tex.print(",handout")
},t]{beamer}
\usepackage[no-math]{luatexja-fontspec}
\usepackage{fontawesome}
\usefonttheme[onlymath]{serif}
\usepackage{luacode}
\input{../template/beamertemp}
\usepackage{ulem}
\defaultfontfeatures{Ligatures={NoRequired, NoCommon, NoContextual}}
\defaultjfontfeatures{Ligatures={NoRequired, NoCommon, NoContextual}}
\setsansfont[Script=Default,Kerning=On,BoldFont=GenShinGothic-Bold,ItalicFont=NotoSansDisplay-LightItalic]{GenShinGothic-Light}
\setmainjfont[%
Script=Default,%
Kerning=On,%
CharacterWidth=AlternateProportional,%
BoldFont=GenShinGothic-P-Bold,%
YokoFeatures={JFM=prop}]{GenShinGothic-P-Light}
\setmonofont[Kerning=Reset]{Inconsolata}
\usepackage{amsmath,amssymb,stmaryrd}
\usepackage{tikz,pgfplots}
\usepackage{url,hyperref}

% \usepackage[style=alphabetic,minalphanames=3,backend=biber]{biblatex}
\renewcommand*{\labelalphaothers}{\textsuperscript{+}}
\addbibresource{main.bib}
\beamertemplatetextbibitems
%% bib underline breaklines https://tex.stackexchange.com/questions/200997/underlined-titles-in-bibliography-with-biblatex-and-ulem-packages
\renewbibmacro*{title}{%
  \ifboolexpr{
    test {\iffieldundef{title}}
    and
    test {\iffieldundef{subtitle}}
  }
    {}
    {\printtext{%
     \printtext[titlecase]{\usefield{\uline}{title}}%
     \setunit{\subtitlepunct}%
     \printfield[titlecase]{subtitle}}%
     \newunit}%
  \printfield{titleaddon}}



\usepackage{graphicx}
\usepackage{xcolor}
\usepackage[most]{tcolorbox}
\newtcolorbox{newbie}{
    enhanced,
    boxrule=0pt,
    right skip=0pt,
    left skip=0pt,
    boxsep=0pt,
    right=.3em,
    left=.3em,
    sharp corners,
    colback=black!20,
}
\usetikzlibrary{shapes,snakes,arrows.meta,positioning,calc,fit,shapes.callouts}
\tikzstyle{every picture} +=[remember picture]

\usepackage{bxcoloremoji}

\def\green#1{\textcolor{green!70!black}{#1}}
\def\blue#1{\textcolor{blue!70!black}{#1}}
\def\yellow#1{\textcolor{orange!70!yellow}{#1}}

% \lstset{
	basicstyle=\ttfamily\small,
	commentstyle=\color{gray},
	keywordstyle=\color{orange},
	stringstyle=\color{purple},
	showstringspaces=false,
	frame=single,
	breaklines=true,
	columns=flexible
}


%% for beamer not to underline
\def\emph#1{\textit{#1}}

\begin{luacode*}
function coordinate(name, xfrom, xto, ypt)
    local xpt = xfrom
    local length = xto - xfrom
    tex.print(([=[ \coordinate (%s) at (%s, %s); ]=]):format(name, xpt, ypt));
    tex.print(([=[ \coordinate[right = %s of %s] (%s'); ]=]):format(length, name, name))
end

function draw(pt, msg, pos)
    pos = pos or "0.5"
    tex.print(([=[ \node at ($(%s)!%s!(%s')$) {%s}; ]=]):format(pt, pos, pt, msg))
    tex.print(([=[ \path (%s) edge (%s'); ]=]):format(pt, pt))
end
\end{luacode*}

\newcommand{\mynode}[3][0.5]{
    \directlua{draw(\asluastring{#2}, \asluastring{#3}, \asluastring{#1})}
}


% \AtBeginSubsection{}

%%%%
\title{研究2017}
\author{河原 悟}
\institute{研究室紹介 2018}
\date{October 9, 2018}
%%%%

\begin{document}
\maketitle

\notoc

\switchfooter
\switchheader
\begin{frame}[c]
    \begin{toccol}
        \tableofcontents
    \end{toccol}
\end{frame}
\switchfooter
\switchheader

\section{自己紹介}
\begin{frame}
    \frametitlesec

    \begin{itemize}
        \item M1
        \item 研究 $\cdots\cdots$ Algebraic Effectsやコントロールオペレータ、言語処理系など
        \item バックグラウンド

            \begin{itemize}
                \item プログラミングは大学に入ってから
                \item 言語処理系が趣味
                \item Luaの処理系や最適化器を作ったりした
            \end{itemize}
    \end{itemize}
\end{frame}

\section{卒業研究}
\begin{frame}
    \frametitlesec

    明示的な合流点を持つコンパイラ中間言語の設計および実装

    \begin{itemize}
        \item コンパイラの実装
        \item 最適化器の実装
        \item 実験(ベンチマーク、比較)
        \item 論文
    \end{itemize}
\end{frame}

\section{工程}
\begin{frame}
    \frametitlesec

    \vskip-1.9\baselineskip

    \begin{figure}[ht]
        \centering
        \begin{tikzpicture}
            \foreach \x in {5,...,12}
                \path[dashed] (\x, 0) edge (\x, -6);

            \foreach \x in {6,...,12}
                \node at (\x, .4) {\x};

            \node at (5, .4) {\textasciitilde 5};

            \foreach \x in {1,...,3}
                \path[dashed] ($(\x, 0) + (12, 0)$) edge ($(\x, -6) + (12, 0)$);

            \foreach \x in {1,...,2}
                \node at ($(\x, .4) + (12, 0)$) {\x};


            \node at (13, 1.2) {\small 2018};
            \node at (15, .4) {3月};

            \directlua{
                coordinate("rinko", 4, 6, "-.7")
                coordinate("impl", 8, 14.5, "-3")
                coordinate("engA", 7.5, 8.1, "-5.2")
                coordinate("inshi", 7.2, 8.5, "-.7")
                coordinate("yomi", 6, 15.5, "-4")
                coordinate("kaki", 11.5, 13.8, "-2.3")
                coordinate("happyo", 13.8, 14.5, "-1.5")
                coordinate("engB", 13.8, 14.5, "-5.2")
                coordinate("poster", 14.5, 15.5, "-.7")
            }

            \begin{scope}[every node/.style={above,fill=white},every path/.style={->,line width=1.5pt}]
                \mynode{rinko}{輪講}
                \mynode[0.3]{impl}{実装}
                \mynode{engA}{専門英語A}
                \mynode{engB}{専門英語B}
                \mynode{inshi}{院試}
                \mynode{yomi}{文献調査}
                \mynode{kaki}{卒論執筆}
                \mynode{happyo}{卒研発表}
                \mynode{poster}{ポスター}
            \end{scope}
        \end{tikzpicture}
    \end{figure}
\end{frame}

\section{やってよかった}
\begin{frame}
    \frametitlesec

    \begin{itemize}
        \item 論文に慣れる

            \begin{itemize}
                \item 読み方
                \item ターミノロジー
                \item 書き方 (\LaTeX の書き方とかも!!)
            \end{itemize}
        \item 開発に慣れる

            研究のPoCがあると良い

            \begin{itemize}
                \item 開発環境
                \item ビルドツール
            \end{itemize}
        \item \coloremoji{💪}体力

            ``研究は体力だ''で検索\coloremoji{🔍}
    \end{itemize}
\end{frame}
\end{document}

