\documentclass[unicode,compress,14pt,CJK%
\directlua{
    handout = os.getenv"HANDOUT"
    local _ = handout and tex.print(",handout")
},t]{beamer}
\usepackage[no-math]{luatexja-fontspec}
\usepackage{fontawesome}
\usefonttheme[onlymath]{serif}
\usepackage{luacode}
\input{../template/beamertemp}
\usepackage{ulem}
\defaultfontfeatures{Ligatures={NoRequired, NoCommon, NoContextual}}
\defaultjfontfeatures{Ligatures={NoRequired, NoCommon, NoContextual}}
\setsansfont[Script=Default,Kerning=On,BoldFont=GenShinGothic-Bold,ItalicFont=NotoSansDisplay-LightItalic]{GenShinGothic-Light}
\setmainjfont[%
Script=Default,%
Kerning=On,%
CharacterWidth=AlternateProportional,%
BoldFont=GenShinGothic-P-Bold,%
YokoFeatures={JFM=prop}]{GenShinGothic-P-Light}
\setmonofont[Kerning=Reset]{Inconsolata}
\usepackage{amsmath,amssymb,stmaryrd}
\usepackage{tikz,pgfplots}
\usepackage{url,hyperref}

% \usepackage[style=alphabetic,minalphanames=3,backend=biber]{biblatex}
\renewcommand*{\labelalphaothers}{\textsuperscript{+}}
\addbibresource{main.bib}
\beamertemplatetextbibitems
%% bib underline breaklines https://tex.stackexchange.com/questions/200997/underlined-titles-in-bibliography-with-biblatex-and-ulem-packages
\renewbibmacro*{title}{%
  \ifboolexpr{
    test {\iffieldundef{title}}
    and
    test {\iffieldundef{subtitle}}
  }
    {}
    {\printtext{%
     \printtext[titlecase]{\usefield{\uline}{title}}%
     \setunit{\subtitlepunct}%
     \printfield[titlecase]{subtitle}}%
     \newunit}%
  \printfield{titleaddon}}



\usepackage{graphicx}
\usepackage{xcolor}
\usepackage[most]{tcolorbox}
\newtcolorbox{newbie}{
    enhanced,
    boxrule=0pt,
    right skip=0pt,
    left skip=0pt,
    boxsep=0pt,
    right=.3em,
    left=.3em,
    sharp corners,
    colback=black!20,
}
\usetikzlibrary{shapes,snakes,arrows.meta,positioning,calc,fit,shapes.callouts}
\tikzstyle{every picture} +=[remember picture]

\usepackage{bxcoloremoji}

\def\green#1{\textcolor{green!70!black}{#1}}
\def\blue#1{\textcolor{blue!70!black}{#1}}
\def\yellow#1{\textcolor{orange!70!yellow}{#1}}

% \lstset{
	basicstyle=\ttfamily\small,
	commentstyle=\color{gray},
	keywordstyle=\color{orange},
	stringstyle=\color{purple},
	showstringspaces=false,
	frame=single,
	breaklines=true,
	columns=flexible
}


%% for beamer not to underline
\def\emph#1{\textit{#1}}

% \AtBeginSubsection{}

\notoc

%%%%
\title{研究2017}
\author{河原 悟}
\institute{研究室紹介 2018}
%%%%

\begin{document}
\maketitle

\section{自己紹介}
\begin{frame}
    \frametitlesec

    \begin{itemize}
        \item M1
        \item 研究 $\cdots\cdots$ Algebraic Effectsに関する研究
        \item バックグラウンド

            \begin{itemize}
                \item プログラミングは大学に入ってから
                \item 言語処理系が趣味
                \item Luaの処理系や最適化器を作ったりした
            \end{itemize}
    \end{itemize}
\end{frame}

\section{卒業研究}
\begin{frame}
    \frametitlesec

    明示的な合流点を持つコンパイラ中間言語の設計および実装

    \begin{itemize}
        \item コンパイラの実装
        \item 最適化器の実装
        \item 実験(ベンチマーク、比較)
        \item 論文
    \end{itemize}
\end{frame}

\section{工程}
\begin{frame}
    \frametitlesec


    \begin{figure}[ht]
        \centering
        \begin{tikzpicture}
            \foreach \x in {5,...,12}
                \path[dashed] (\x, 0) edge (\x, -3);

            \foreach \x in {6,...,12}
                \node at (\x, .5) {\x};

            \node at (5, .5) {\textasciitilde 5};

            \foreach \x in {1,...,3}
                \path[dashed] ($(\x, 0) + (12, 0)$) edge ($(\x, -3) + (12, 0)$);

            \foreach \x in {1,...,2}
                \node at ($(\x, .5) + (12, 0)$) {\x};


            \node at (15, .5) {3月};

            \path[->,line width=1.5pt]
                (4, -.3) edge (6, -.3) % 輪講など
                (8, -1.8) edge (15.5, -1.8) % 実装
                (7.5, -2.2) edge (8.1, -2.2) % 英語A
                (13.8, -2.2) edge (14.5, -2.2) % 英語B
                (7.2, -2.7) edge (8.5, -2.7) % 院試
                (6, -1.4) edge (15.5, -1.4) % 論文読み
                (10, -.5) edge (10.7, -.5) % 中間
                (11.5, -1) edge (13.8, -1) % 論文かき
                (13.8, -1) edge (14.5, -1) % 発表
                (14.5, -1) edge (15.5, -1) % ポスター
                ;
        \end{tikzpicture}
    \end{figure}
\end{frame}
\end{document}

