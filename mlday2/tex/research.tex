\section{研究紹介}
\begin{frame}
    \frametitlesec

    \structure{$\bullet$} \green{Asymmetric Coroutine}による\alert{Oneshot Algebraic Effects}の実装

    \begin{center}
        \small
        \begin{tikzpicture}
            \node[ellipse,draw] (oneshotsr) {\structure{oneshot shift/reset}};
            \node[ellipse,draw, below left = of oneshotsr, xshift = 1.7\zw,yshift=.6\zh, text width=4.4\zw] (oneshotae) {\alert{oneshot Algebraic Effects}};
            \node[ellipse,draw, text width=5.5\zw, below right = of oneshotsr, xshift = -2\zw,yshift=.2\zh] (asymcor) {\green{asymmetric coroutine}};

            \path[->, line width=0.07\zw]
                (oneshotae) edge node[above left,sloped]{1} (oneshotsr)
                (oneshotsr) edge node[above right,sloped]{2} (asymcor);
        \end{tikzpicture}
    \end{center}
    \vskip-2\zh

    \begin{itemize}
        \item[\coloremoji{👊}] (asymmetric) coroutineのある言語でAEが使える
        \item[\coloremoji{☝}] 1, 2のそれぞれ既存の変換を参考にする
    \end{itemize}
\end{frame}

\subsection{oneshot AE $\rightarrow$ oneshot s/r}
\begin{frame}[fragile]{\normalsize \alert{oneshot} Algebraic Effects $\rightarrow$ \alert{oneshot} shift/reset}

    \vskip.5\baselineskip

    \cite{kiselyov2016eff}に基づき\structure{\hyperlink{fr:delimcc}{先程の実装}}の逆をやる

    \begin{lstlisting}[basicstyle=\scriptsize\ttfamily\color{green!60!black},xleftmargin=4\zw,xrightmargin=4\zw]
module Translate(D: DELIMCC) : sig
  type ('a, 'b) free
  type 'a thunk = unit -> 'a
  val newi : unit -> 'a D.prompt
  val op : ('a, 'b) free D.prompt -> 'a -> 'b
  val handler : ('g, 'g) free D.prompt ->
    ('g -> 'o) -> ('g * ('g -> 'o) -> 'o) -> 'g thunk -> 'o
  val handle : ('a thunk -> 'b) -> 'a thunk -> 'b
end = struct
  ......
end
    \end{lstlisting}

    \bgroup
    \small
    \begin{itemize}
        \item[Q.] 継続のoneshotnessをAEが引き継いでることの証明は?
        \item[A.] まだ無い。affine typeで型によって示す予定
    \end{itemize}
    \egroup
\end{frame}

\subsection{oneshot s/r $\rightarrow$ AC}
\begin{frame}[fragile]{\normalsize \alert{oneshot} shift/reset $\rightarrow$ Asymmetric Coroutine}


    \vskip.5\baselineskip

    \cite{UsuiMaster2017}による変換を使う (Luaによる実装)

    \begin{multicols}{2}
        \begin{lstlisting}[basicstyle=\scriptsize\ttfamily\color{green!60!black},language={[5.3]lua}]
function sr.reset(th)
  local l = coro.create(
    function(_)
      return (function(y)
        return function(_)
          return y
        end
      end)(th())
    end)

  return coro.resume(l)()
end

function sr.shift(f)
  local k = coro.current

  return coro.yield(function(_)
    return sr.reset(function()
      return f(k)
    end)
  end)
end

function sr.throw(k, e)
  return coro.resume(k, e)()
end
        \end{lstlisting}
    \end{multicols}
\end{frame}

\subsection{課題、今後の予定}
\begin{frame}
    \frametitlesubs

    \begin{itemize}
        \item oneshotnessの保証(先述)
        \item Usuiによる変換の対象のshift/resetをmulti-promptで再定義
        \item oneshot shift/resetを経由しない\yellow{ダイレクトな変換}を考える
    \end{itemize}

\begin{center}
        \small
        \begin{tikzpicture}
            \node[ellipse,draw] (oneshotsr) {\structure{oneshot shift/reset}};
            \node[ellipse,draw, below left = of oneshotsr, xshift = 1.7\zw,yshift=.6\zh, text width=4.4\zw] (oneshotae) {\alert{oneshot Algebraic Effects}};
            \node[ellipse,draw, text width=5.5\zw, below right = of oneshotsr, xshift = -2\zw,yshift=-.0\zh] (asymcor) {\green{asymmetric coroutine}};

            \path[->, line width=0.07\zw]
                (oneshotae) edge (oneshotsr)
                (oneshotsr) edge (asymcor);

            \path[->,line width=.15\zw,color=orange!70!yellow] (oneshotae) edge (asymcor);
        \end{tikzpicture}
    \end{center}
\end{frame}

