\section{Algebraic Effects}
\newsavebox{\lstbox}
\begin{lrbox}{\lstbox}
    \begin{lstlisting}
!\tikz \node[coordinate] (eff0) {};\null!effect Say   : string -> unit !\tikz \node[coordinate] (effdef) {};!
!\null!effect Twice : unit   -> unit!\tikz \node[coordinate] (effe) {};!

!\tikz \node[coordinate] (handle0) {};\null!handle (
  !\tikz \node[coordinate] (effinvoke) {};\null!perform (Say "Hello");
  !\null!perform (Twice ());
  !\null!perform (Say "world")
) !\tikz \node[coordinate] (handlew) {};\null!with
!\tikz \node[coordinate] (effh0) {};\null!| x -> x!\tikz \node[coordinate] (handler) {};!
| effect (Twice ()) !\tikz \node[coordinate] (cont) {};\null!k -> k (); k ()
| effect (Say msg)  !\null!k ->
    print_endline msg; k ()!\tikz \coordinate (effhe) {};!
    \end{lstlisting}
\end{lrbox}

\begin{frame}[fragile]
    \frametitlesec

    イメージとしては\textbf{\alert{継続}を持てる\structure{例外}}\pause

    例(Eff language implementation):

    \usebox{\lstbox}
    \pause
    \begin{tikzpicture}[remember picture, overlay]
        \only<+->{
            \node[ellipse callout, text=blue, draw=blue, line width = 1pt, fill = white,
              callout absolute pointer={($(effdef) - (.2, 1)$)}] at ($(effdef) + (2.1, .7) - (0, 1.4)$) {\small effect definition};
            \fill[fill opacity=0.2,blue] ($(eff0) + (0, -.9)$) rectangle ($(effe)+(.05, -1.4)$);
        }
        \only<+->{
            \node[ellipse callout, text=orange!70!yellow, draw=orange!70!yellow, line width=1pt, fill=white,
              callout absolute pointer={($(effinvoke) + (5.7, -.1) - (0, .7)$)}] at ($(effinvoke) + (9.0, .7) - (0, .8)$) {\small effect invocation};
            \fill[fill opacity=0.2,orange!70!yellow] ($(effinvoke) + (0, .4) - (0, 1.3)$) rectangle ($(effinvoke)+(5.7, -1.1) - (0, 1.3)$);
        }
        \only<+->{
            \node[ellipse callout, text=green!60!black, draw=green!60!black, line width = 1pt, fill = white,
                callout absolute pointer={($(handler) + (5, .5) -(0,1.3)$)}] at ($(handler) + (7.7, 1.6) -(0,1.1)$) {\small effect handler};
            \fill[fill opacity=0.3,green!70!yellow] ($(handle0) + (0, .4) -(0,1.3)$) rectangle ($(handle0) + (1.55, -.1)-(0,1.3)$);
            \fill[fill opacity=0.3,green!70!yellow] ($(handlew) + (0, .4) -(0,1.3)$) rectangle ($(handlew) + (1.15, -.1)-(0,1.3)$);
            \fill[fill opacity=0.3,green!70!yellow] ($(effh0) + (0, .4)-(0,1.3)$) rectangle ($(effhe) + (2., -.1)-(0,1.3)$);
        }

        \only<+->{
            \node[fill=red, text = white,ellipse,rectangle] at ($(cont)-(0,1.3)+(.1,.2)$) {\small k};
            \node[fill=red, text = white,ellipse,rectangle] at ($(cont)-(0,1.3)+(.1,-.35)$) {\small k};
            \node[ellipse callout, text=red, draw=red, line width = 1pt, fill = white,
                callout absolute pointer={($(cont) -(0,1.3) + (.2, .4)$)}] at ($(cont) -(0,1.3) + (4.4, .8)$) {\small \textbf{continuation}};
        }
    \end{tikzpicture}
\end{frame}

\subsection{effectの型}
\begin{frame}[fragile]
    \frametitlesubs

    \begin{center}
        \begin{minipage}{.7\textwidth}
            \hfill なにこれ{\large \coloremoji{🤔}}

            \begin{lstlisting}[xrightmargin=2.8\zw,xleftmargin=1\zw]
effect Say : !\tikz \node[coordinate] (str) {};\null!string -> !\tikz \node[coordinate] (unit) {};\null!unit
            \end{lstlisting}
            \begin{tikzpicture}[remember picture, overlay]
                \onslide<3->{
                    \lstbasiccolor{blue!60!black}
                    \fill[fill opacity=0.3, blue!60!black] ($(str) + (0, -.15)$) rectangle ($(str) +(1.6, .4)$);
                }

                \onslide<4->{
                    \lstbasiccolor{orange!70!yellow}
                    \fill[fill opacity=0.3, orange!70!yellow] ($(unit)+(0, -.15)$) rectangle ($(unit)+(1.1, .4)$);
                }
            \end{tikzpicture}
        \end{minipage}

        \pause
    \end{center}

    \vskip-\zh%
    \hskip2\zw%
    例:\hskip7\zw

    \begin{center}
        \begin{minipage}{.7\textwidth}
            \begin{lstlisting}[xrightmargin=3.1\zw,xleftmargin=2\zw]
!\tikz \node[coordinate] (ctxt) {};\null $E\left[\right.$\tikz \node[coordinate] (entire) {};\null!perform (Say !\tikz \node[coordinate] (str) {};\null!"Hello")!\tikz \node[coordinate] (ctxte){};\null $\left.\right]$!
            \end{lstlisting}
            \pause
            \begin{tikzpicture}[remember picture, overlay]
                \onslide<3->{
                    \lstbasiccolor{blue!60!black}
                    \node at ($(str) + (.9, -.4)$) {\lstinline{: string}};
                    \fill[fill opacity=0.3, blue!60!black] ($(str)+(0, -.15)$) rectangle ($(str)+(1.8, .4)$);
                }
                \onslide<4->{
                    \lstbasiccolor{orange!70!yellow}
                    \node at ($(entire) + (3, -.8)$) {\lstinline{: unit}};
                    \fill[fill opacity=0.3, orange!70!yellow] ($(entire)+(0, -.6)$) rectangle ($(entire)+(5.3, .4)$);
                }
                \onslide<5->{
                    \lstbasiccolor{red}
                    \node at ($(ctxt)+(3, .6)$) {\lstinline{: unit -> 'a} {\scriptsize (継続の型)}};
                    \fill[fill opacity=0.5, red] ($(ctxt)+(0, -.15)$) rectangle ($(ctxt)+(.52,.4)$);
                    \fill[fill opacity=0.5, red] ($(ctxte)+(0, -.15)$) rectangle ($(ctxte)+(.3,.4)$);
                }
            \end{tikzpicture}
        \end{minipage}
    \end{center}

    \begin{center}
        \vfill
        \onslide<6>{
            {\Huge \textbf{完全に理解した}}
        }
        \vfill
    \end{center}
\end{frame}

\subsection{デモ}
\begin{frame}[fragile]
    \frametitlesubs

    \usebox{\lstbox}
\end{frame}

\subsection{ポイント}
\begin{frame}
    \frametitlesubs

    \begin{flalign*}
        &\begin{array}{lc}
            \left.\begin{array}{rr}
                    \bullet&\text{定義と実装の\textbf{分離}}\\
                    \bullet&\text{ハンドラの\textbf{合成}}
                \end{array}\right\} & \begin{array}{l}
                \text{モジュラーなプログラミング}\\
                \text{を支援}
            \end{array}
        \end{array}\\
        &\begin{array}{lc}
            \left.\begin{array}{rr}
                    \bullet&\text{\textbf{限定継続}が使える}
            \end{array}\right\}&\text{例外より強力}
        \end{array}
    \end{flalign*}
\end{frame}

