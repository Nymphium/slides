\documentclass[unicode,compress,14pt,CJK%
\directlua{
    handout = os.getenv"HANDOUT"
    local _ = handout and tex.print(",handout")
},t]{beamer}
\usepackage[no-math]{luatexja-fontspec}
\usepackage{fontawesome}
\usefonttheme[onlymath]{serif}
\usepackage{luacode}
\input{../template/beamertemp}
\usepackage{ulem}
\defaultfontfeatures{Ligatures={NoRequired, NoCommon, NoContextual}}
\defaultjfontfeatures{Ligatures={NoRequired, NoCommon, NoContextual}}
\setsansfont[Script=Default,Kerning=On,BoldFont=GenShinGothic-Bold,ItalicFont=NotoSansDisplay-LightItalic]{GenShinGothic-Light}
\setmainjfont[%
Script=Default,%
Kerning=On,%
CharacterWidth=AlternateProportional,%
BoldFont=GenShinGothic-P-Bold,%
YokoFeatures={JFM=prop}]{GenShinGothic-P-Light}
\setmonofont[Kerning=Reset]{Inconsolata}
\usepackage{amsmath,amssymb,stmaryrd}
\usepackage{tikz,pgfplots}
\usepackage{url,hyperref}

\usepackage[style=alphabetic,minalphanames=3,backend=biber]{biblatex}
\renewcommand*{\labelalphaothers}{\textsuperscript{+}}
\addbibresource{main.bib}
\beamertemplatetextbibitems
%% bib underline breaklines https://tex.stackexchange.com/questions/200997/underlined-titles-in-bibliography-with-biblatex-and-ulem-packages
\renewbibmacro*{title}{%
  \ifboolexpr{
    test {\iffieldundef{title}}
    and
    test {\iffieldundef{subtitle}}
  }
    {}
    {\printtext{%
     \printtext[titlecase]{\usefield{\uline}{title}}%
     \setunit{\subtitlepunct}%
     \printfield[titlecase]{subtitle}}%
     \newunit}%
  \printfield{titleaddon}}

\usepackage{graphicx}
\usepackage{xcolor}
\usepackage[most]{tcolorbox}
\newtcolorbox{newbie}{
    enhanced,
    boxrule=0pt,
    right skip=0pt,
    left skip=0pt,
    boxsep=0pt,
    right=.3em,
    left=.3em,
    sharp corners,
    colback=black!20,
}
\usetikzlibrary{shapes,snakes,arrows.meta,positioning,calc,fit,shapes.callouts}
\tikzstyle{every picture} +=[remember picture]

% \tikzset{
    % multiline/.style = {text badly ragged, text badly centered},
    % lang/.style = {ellipse, draw},
    % boldarrow/.style = {draw=black,solid,fill=black,line width=.6pt},
% }
%
% \newfontfamily\NotoEmoji{NotoColorEmoji}
% \newfontfamily\NotoEmoji{Twitter Color Emoji}

\usepackage{bxcoloremoji}

\def\green#1{\textcolor{green!70!black}{#1}}
\def\blue#1{\textcolor{blue!70!black}{#1}}
\def\yellow#1{\textcolor{orange!70!yellow}{#1}}

\def\lstbasiccolor#1{
  \lstset{
    basicstyle=\small\ttfamily\color{#1},
    identifierstyle={\ttfamily},
    keywordstyle=[1]{\ttfamily},
    keywordstyle=[7]{\ttfamily},
    keywordstyle=[8]{\bfseries\ttfamily},
    literate=*{->}{$\rightarrow{}$}{1},
  }
}

\makeatletter
\newcommand*\idstyle{%
  \expandafter\id@style\the\lst@token\relax
}
\def\id@style#1#2\relax{%
  \ifcat#1\relax\else
    \ifnum`#1=\uccode`#1%
      \color{magenta}\bfseries\ttfamily
    \fi
  \fi
}
\makeatother

\def\CodeSymbol#1{\textcolor{black!5!blue!50}{\texttt{#1}}}

\lstset{
    language={[Objective]caml},
    escapechar=!,
    numbers=none,
    basicstyle=\small\ttfamily\color{green!60!black},
    backgroundcolor=\color{black!7!yellow!9},
    columns=fixed,
    float=tbpm,
    moredelim={[is][\color{red}\uwave]{<@}{@>}},
    moredelim={[is]{<0@}{@0>}},
    morekeywords=[1]{
        effect,handle
    },
    morekeywords=[8]{
        perform,continue
    },
    morekeywords=[10]{|>},
    literate=*{->}{{{\textcolor{black!5!orange!90}{$\rightarrow$}{}}}}{1}
             {\{}{{\CodeSymbol{\{}}}1
             {\}}{{\CodeSymbol{\}}}}1
             {(}{{\CodeSymbol{(}}}1
             {)}{{\CodeSymbol{)}}}1
             {|}{{\CodeSymbol{|}}}1
             {|>}{{$\mid >$}}1
             {;}{{\CodeSymbol{;}}}1
             {\_}{\_}1
             {:}{{\CodeSymbol{:}}}1,
    keywordstyle=[1]{\color{black!5!orange!90}\ttfamily},
    keywordstyle=[7]{\color{black!5!blue!90}\ttfamily},
    keywordstyle=[8]{\color{black!5!cyan!80}\bfseries\ttfamily},
    identifierstyle=\idstyle,
}

\def\emph#1{\textit{#1}}

\AtBeginSubsection{}
% \tocmonocoltrue

\title{\textcolor{black}{Dive into \textit{Algebraic Effects}}}
\author{びしょ〜じょ}
\date{September 16, 2018}
\institute{ML Days \#2}

\begin{document}
\maketitle

\switchfooter
\begin{frame}{やること}
    \vfill
    \begin{itemize}
        \item[\faicon{wifi}] \alert{Algebraic Effects}を伝道

            \begin{itemize}
                \item[\coloremoji{🤔}] Algebraic Effects is 何
                \item[\blue{\faicon{language}}] Algebraic Effectsが使える言語
                \item[\coloremoji{🙆}] Algebraic Effectsの活用事例
            \end{itemize}
        \item[\coloremoji{🔬}] 研究のご紹介

            {\small 先日JSSSTでポスター発表した内容を紹介シマス}
    \end{itemize}
    \vfill
\end{frame}

\begin{frame}{目次}
    \tableofcontents
\end{frame}
\switchfooter

\section{自己紹介}
\begin{frame}
    \frametitlesec

    \vfill

    \begin{minipage}[b]{.1\textwidth}
        \centering
        \raisebox{-.5\zh}{\includegraphics[height=2\zh]{img/yh.png}}
    \end{minipage}
    \begin{minipage}[b]{.49\textwidth}
        こんにちは、びしょ〜じょです。
    \end{minipage}

    \begin{itemize}
        \item T大学大学院でM1をやっている
        \item プログラム言語とかの研究
        \item 少し前はLuaにお熱だった
    \end{itemize}

    \begin{itemize}
        \item[\textcolor{blue!60!white}{\faicon{twitter}}] \href{https://twitter.com/Nymphium}{Nymphium}
        \item[\textcolor{black}{\faicon{github}}] \href{https://github.com/nymphium}{Nymphium}
    \end{itemize}
    \vfill
\end{frame}

\section{Algebraic Effects}
\newsavebox{\lstbox}
\begin{lrbox}{\lstbox}
    \begin{lstlisting}
!\tikz \node[coordinate] (eff0) {};\null!effect Say   : string -> unit !\tikz \node[coordinate] (effdef) {};!
!\null!effect Twice : unit   -> unit!\tikz \node[coordinate] (effe) {};!

!\tikz \node[coordinate] (handle0) {};\null!handle (
  !\tikz \node[coordinate] (effinvoke) {};\null!perform (Say "Hello");
  !\null!perform (Twice ());
  !\null!perform (Say "world")
) !\tikz \node[coordinate] (handlew) {};\null!with
!\tikz \node[coordinate] (effh0) {};\null!| x -> x!\tikz \node[coordinate] (handler) {};!
| effect (Twice ()) !\tikz \node[coordinate] (cont) {};\null!k -> k (); k ()
| effect (Say msg)  !\null!k ->
    print_endline msg; k ()!\tikz \coordinate (effhe) {};!
    \end{lstlisting}
\end{lrbox}

\begin{frame}[fragile]
    \frametitlesec

    イメージとしては\textbf{\alert{継続}を持てる\structure{例外}}\pause

    例(Eff language implementation):

    \usebox{\lstbox}
    \pause
    \begin{tikzpicture}[remember picture, overlay]
        \only<+->{
            \node[ellipse callout, text=blue, draw=blue, line width = 1pt, fill = white,
              callout absolute pointer={($(effdef) - (.2, 1)$)}] at ($(effdef) + (2.1, .7) - (0, 1.4)$) {\small effect definition};
            \fill[fill opacity=0.2,blue] ($(eff0) + (0, -.9)$) rectangle ($(effe)+(.05, -1.4)$);
        }
        \only<+->{
            \node[ellipse callout, text=orange!70!yellow, draw=orange!70!yellow, line width=1pt, fill=white,
              callout absolute pointer={($(effinvoke) + (5.7, -.1) - (0, .7)$)}] at ($(effinvoke) + (9.0, .7) - (0, .8)$) {\small effect invocation};
            \fill[fill opacity=0.2,orange!70!yellow] ($(effinvoke) + (0, .4) - (0, 1.3)$) rectangle ($(effinvoke)+(5.7, -1.1) - (0, 1.3)$);
        }
        \only<+->{
            \node[ellipse callout, text=green!60!black, draw=green!60!black, line width = 1pt, fill = white,
                callout absolute pointer={($(handler) + (5, .5) -(0,1.3)$)}] at ($(handler) + (7.7, 1.6) -(0,1.1)$) {\small effect handler};
            \fill[fill opacity=0.3,green!70!yellow] ($(handle0) + (0, .4) -(0,1.3)$) rectangle ($(handle0) + (1.55, -.1)-(0,1.3)$);
            \fill[fill opacity=0.3,green!70!yellow] ($(handlew) + (0, .4) -(0,1.3)$) rectangle ($(handlew) + (1.15, -.1)-(0,1.3)$);
            \fill[fill opacity=0.3,green!70!yellow] ($(effh0) + (0, .4)-(0,1.3)$) rectangle ($(effhe) + (2., -.1)-(0,1.3)$);
        }

        \only<+->{
            \node[fill=red, text = white,ellipse,rectangle] at ($(cont)-(0,1.3)+(.1,.2)$) {\small k};
            \node[fill=red, text = white,ellipse,rectangle] at ($(cont)-(0,1.3)+(.1,-.35)$) {\small k};
            \node[ellipse callout, text=red, draw=red, line width = 1pt, fill = white,
                callout absolute pointer={($(cont) -(0,1.3) + (.2, .4)$)}] at ($(cont) -(0,1.3) + (4.4, .8)$) {\small \textbf{continuation}};
        }
    \end{tikzpicture}
\end{frame}

\subsection{effectの型}
\begin{frame}[fragile]
    \frametitlesubs

    \begin{center}
        \begin{minipage}{.7\textwidth}
            \hfill なにこれ{\large \coloremoji{🤔}}

            \begin{lstlisting}[xrightmargin=2.8\zw,xleftmargin=1\zw]
effect Say : !\tikz \node[coordinate] (str) {};\null!string -> !\tikz \node[coordinate] (unit) {};\null!unit
            \end{lstlisting}
            \begin{tikzpicture}[remember picture, overlay]
                \onslide<3->{
                    \lstbasiccolor{blue!60!black}
                    \fill[fill opacity=0.3, blue!60!black] ($(str) + (0, -.15)$) rectangle ($(str) +(1.6, .4)$);
                }

                \onslide<4->{
                    \lstbasiccolor{orange!70!yellow}
                    \fill[fill opacity=0.3, orange!70!yellow] ($(unit)+(0, -.15)$) rectangle ($(unit)+(1.1, .4)$);
                }
            \end{tikzpicture}
        \end{minipage}

        \pause
    \end{center}

    \vskip-\zh%
    \hskip2\zw%
    例:\hskip7\zw

    \begin{center}
        \begin{minipage}{.7\textwidth}
            \begin{lstlisting}[xrightmargin=3.1\zw,xleftmargin=2\zw]
!\tikz \node[coordinate] (ctxt) {};\null $E\left[\right.$\tikz \node[coordinate] (entire) {};\null!perform (Say !\tikz \node[coordinate] (str) {};\null!"Hello")!\tikz \node[coordinate] (ctxte){};\null $\left.\right]$!
            \end{lstlisting}
            \pause
            \begin{tikzpicture}[remember picture, overlay]
                \onslide<3->{
                    \lstbasiccolor{blue!60!black}
                    \node at ($(str) + (.9, -.4)$) {\lstinline{: string}};
                    \fill[fill opacity=0.3, blue!60!black] ($(str)+(0, -.15)$) rectangle ($(str)+(1.8, .4)$);
                }
                \onslide<4->{
                    \lstbasiccolor{orange!70!yellow}
                    \node at ($(entire) + (3, -.8)$) {\lstinline{: unit}};
                    \fill[fill opacity=0.3, orange!70!yellow] ($(entire)+(0, -.6)$) rectangle ($(entire)+(5.3, .4)$);
                }
                \onslide<5->{
                    \lstbasiccolor{red}
                    \node at ($(ctxt)+(3, .6)$) {\lstinline{: unit -> 'a} {\scriptsize (継続のanswer type)}};
                    \fill[fill opacity=0.5, red] ($(ctxt)+(0, -.15)$) rectangle ($(ctxt)+(.52,.4)$);
                    \fill[fill opacity=0.5, red] ($(ctxte)+(0, -.15)$) rectangle ($(ctxte)+(.3,.4)$);
                }
            \end{tikzpicture}
        \end{minipage}
    \end{center}

    \begin{center}
        \vfill
        \onslide<6>{
            {\Huge \textbf{完全に理解した}}
        }
        \vfill
    \end{center}
\end{frame}

\subsection{デモ}
\begin{frame}[fragile]
    \frametitlesubs

    \usebox{\lstbox}
\end{frame}

\subsection{ポイント}
\begin{frame}
    \frametitlesubs

    \begin{flalign*}
        &\begin{array}{lc}
            \left.\begin{array}{rr}
                    \bullet&\text{定義と実装の\textbf{分離}}\\
                    \bullet&\text{ハンドラの\textbf{合成}}
                \end{array}\right\} & \begin{array}{l}
                \text{モジュラーなプログラミング}\\
                \text{を支援}
            \end{array}
        \end{array}\\
        &\begin{array}{lc}
            \left.\begin{array}{rr}
                    \bullet&\text{\textbf{限定継続}が使える}
            \end{array}\right\}&\text{例外より強力}
        \end{array}
    \end{flalign*}
\end{frame}

\section{Algebraic Effectsが使える言語}
\begin{frame}[fragile]
    \frametitlesec

    \begin{itemize}
        \item Eff\footnote{\url{https://www.eff-lang.org/}}

            \begin{itemize}
                \item MLベースのsyntax、型推論
                \item 処理系やライブラリ(EDSL)など実装多数
            \end{itemize}

        \item Koka\footnote{\url{https://koka-lang.github.io/koka/}}

            \begin{itemize}
                \item MS Research製
                \item effectが型で表される

                    (ex: \lstinline{println : <@a -> console ()@>})
                \item Algebraic Effects以外にもおもしろ機能
            \end{itemize}

        \item Multicore \textbf{OCaml}\footnote{\url{http://ocamllabs.io/doc/multicore.html}}

            \begin{itemize}
                \item OCaml LabsがOCamlをfork
                \item 継続が\alert{oneshot}のAlgebraic Effetsを持つ
            \end{itemize}
    \end{itemize}
\end{frame}

\subsection{Multicore OCaml}
\begin{frame}[fragile]
    \frametitlesubs

    {\small Stateモナド風}

    \begin{multicols}{2}
        \begin{lstlisting}[basicstyle=\scriptsize\ttfamily\color{green!60!black}]
module State(S : sig type t end)
= struct
  type t = S.t

  effect Put : t -> unit;;
  effect Get : t

  let run init f =
    init |> match f () with
    | x -> (fun s -> (s, x))
    | effect (Put s') k ->
      (fun s -> continue k () s')
    | effect Get k ->
      (fun s -> continue k s s)
end

effect Log : int -> unit
let log msg = perform @@ Log msg
try begin
  let module S1 =
    State(struct
      type t = int
    end) in
  let open S1 in
  let incr () = 
   perform (Put (perform Get + 1))
  in
  run 0 @@ fun () ->
      incr ();
      log @@ perform Get;
      incr ();
      log @@ perform Get
end
with effect (Log msg) k ->
  print_int msg; continue k ()
        \end{lstlisting}
    \end{multicols}

    \vfill
\end{frame}

\hypertarget{fr:delimcc}{}
\label{fr:deimcc}
\begin{frame}[fragile]
    \frametitlesubs

    shift/resetも実装できるぞ!!
    \phantomsection
    \begin{lstlisting}[basicstyle=\scriptsize\ttfamily\color{green!60!black}]
module DelimccOne = struct
  type 'a prompt = {
    take : 'b. (('b -> 'a) -> 'a) -> 'b;
    push : (unit -> 'a) -> 'a;
  }

  let new_prompt (type a) : unit -> a prompt = fun () ->
    let module M = struct
      effect Prompt : (('b -> a) -> a) -> 'b
    end in
    let take f = perform (M.Prompt f) in
    let push th =
      try th () with effect (M.Prompt f) k -> f @@ continue k
    in
    {take; push}

  let push_prompt {push} = push
  let take_subcont {take} = take

  let shift0 p f = take_subcont p @@ fun k -> f k
end
    \end{lstlisting}
\end{frame}

\section{研究紹介}
\begin{frame}
    \frametitlesec

    \structure{$\bullet$} \green{Asymmetric Coroutine}による\alert{Oneshot Algebraic Effects}の実装

    \begin{center}
        \small
        \begin{tikzpicture}
            \node[ellipse,draw] (oneshotsr) {\structure{oneshot shift/reset}};
            \node[ellipse,draw, below left = of oneshotsr, xshift = 1.7\zw,yshift=.6\zh, text width=4.4\zw] (oneshotae) {\alert{oneshot Algebraic Effects}};
            \node[ellipse,draw, text width=5.5\zw, below right = of oneshotsr, xshift = -2\zw,yshift=.2\zh] (asymcor) {\green{asymmetric coroutine}};

            \path[->, line width=0.07\zw]
                (oneshotae) edge node[above left,sloped]{1} (oneshotsr)
                (oneshotsr) edge node[above right,sloped]{2} (asymcor);
        \end{tikzpicture}
    \end{center}
    \vskip-2\zh

    \begin{itemize}
        \item[\coloremoji{👊}] (asymmetric) coroutineのある言語でAEが使える
        \item[\coloremoji{☝}] 1, 2のそれぞれ既存の変換を参考にする
    \end{itemize}
\end{frame}

\subsection{oneshot AE $\rightarrow$ oneshot s/r}
\begin{frame}[fragile]{\normalsize \alert{oneshot} Algebraic Effects $\rightarrow$ \alert{oneshot} shift/reset}

    \vskip.5\baselineskip

    \cite{kiselyov2016eff}に基づき\structure{\hyperlink{fr:delimcc}{先程の実装}}の逆をやる

    \begin{lstlisting}[basicstyle=\scriptsize\ttfamily\color{green!60!black},xleftmargin=4\zw,xrightmargin=4\zw]
module Translate(D: DELIMCC) : sig
  type ('a, 'b) free
  type 'a thunk = unit -> 'a
  val newi : unit -> 'a D.prompt
  val op : ('a, 'b) free D.prompt -> 'a -> 'b
  val handler : ('g, 'g) free D.prompt ->
    ('g -> 'o) -> ('g * ('g -> 'o) -> 'o) -> 'g thunk -> 'o
  val handle : ('a thunk -> 'b) -> 'a thunk -> 'b
end = struct
  ......
end
    \end{lstlisting}

    \bgroup
    \small
    \begin{itemize}
        \item[Q.] 継続のoneshotnessをAEが引き継いでることの証明は?
        \item[A.] まだ無い。affine typeで型によって示す予定
    \end{itemize}
    \egroup
\end{frame}

\subsection{oneshot s/r $\rightarrow$ AC}
\begin{frame}[fragile]{\normalsize \alert{oneshot} shift/reset $\rightarrow$ Asymmetric Coroutine}


    \vskip.5\baselineskip

    \cite{UsuiMaster2017}による変換を使う (Luaによる実装)

    \begin{multicols}{2}
        \begin{lstlisting}[basicstyle=\scriptsize\ttfamily\color{green!60!black},language={[5.3]lua}]
function sr.reset(th)
  local l = coro.create(
    function(_)
      return (function(y)
        return function(_)
          return y
        end
      end)(th())
    end)

  return coro.resume(l)()
end

function sr.shift(f)
  local k = coro.current

  return coro.yield(function(_)
    return sr.reset(function()
      return f(k)
    end)
  end)
end

function sr.throw(k, e)
  return coro.resume(k, e)()
end
        \end{lstlisting}
    \end{multicols}
\end{frame}

\subsection{課題、今後の予定}
\begin{frame}
    \frametitlesubs

    \begin{itemize}
        \item oneshotnessの保証(先述)
        \item Usuiによる変換の対象のshift/resetをmulti-promptで再定義
        \item oneshot shift/resetを経由しない\yellow{ダイレクトな変換}を考える
    \end{itemize}

\begin{center}
        \small
        \begin{tikzpicture}
            \node[ellipse,draw] (oneshotsr) {\structure{oneshot shift/reset}};
            \node[ellipse,draw, below left = of oneshotsr, xshift = 1.7\zw,yshift=.6\zh, text width=4.4\zw] (oneshotae) {\alert{oneshot Algebraic Effects}};
            \node[ellipse,draw, text width=5.5\zw, below right = of oneshotsr, xshift = -2\zw,yshift=-.0\zh] (asymcor) {\green{asymmetric coroutine}};

            \path[->, line width=0.07\zw]
                (oneshotae) edge (oneshotsr)
                (oneshotsr) edge (asymcor);

            \path[->,line width=.15\zw,color=orange!70!yellow] (oneshotae) edge (asymcor);
        \end{tikzpicture}
    \end{center}
\end{frame}

\section{まとめ}
\begin{frame}
    \frametitlesec

    \begin{itemize}
        \item[\coloremoji{🌋}] Algebraic Effectsが楽しい

            ICFP 2018やML Workshop2018にも\\
            AE関連のトピック%
            \footnote{\url{https://icfp18.sigplan.org/event/mlfamilyworkshop-2018-papers-programming-with-abstract-algebraic-effects}}%
            \footnote{\url{https://icfp18.sigplan.org/event/icfp-2018-papers-versatile-event-correlation-with-algebraic-effects}}

        \item[\coloremoji{🉐}] Algebraic Effects使おう

            \begin{itemize}
                \item[\coloremoji{🛠}] インプリいろいろ
                \item[\coloremoji{💪}] なければ自作も可
            \end{itemize}

        \item[\coloremoji{👨‍💻}] 研究やってます
    \end{itemize}
\end{frame}

\nofootertrue
\begin{frame}
    \vfill
    \centering
    {\Huge \textbf{おわり}}
    \vfill
\end{frame}

    \nocite{*}
\begin{frame}{参考文献}
    \scriptsize
    \printbibliography[heading=none]
\end{frame}
\end{document}
