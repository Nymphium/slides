\section{The Evolution of Continuations}
\TableOfContents{}

\subsection{Dump and J operator, SECD machine}
\begin{frame}[fragile]
	\frametitlesubs

	In 1960s, the first abstract machine for evaluating \textcolor{subhighlight}{functional} programs\cite{landin1964mechanical}:

	\only<.(3)->\small
	\vspace{-.4\zh}
	\begin{table}[h!]
		\begin{tabular}{lcl}

			{\boldslant S}\texttt{tack}                   & : & Stores \underline{intermediate results}                                                \\
			{\boldslant E}\texttt{nvironment}             & : & Stores \underline{variables}                                                           \\
			{\boldslant C}\texttt{ontrol}                 & : & Stores \underline{the next instruction}                                                \\

			\only<.(3)->\large {\boldslant D}\texttt{ump} & : & Stores {\only<.(3)->{\large\bf\color{highlight}}\underline{the suspended computation}} \\
		\end{tabular}
	\end{table}

	\pause
	\vspace{-.4\zh}
	\hspace{1.7\zw}
	And
	\only<.(2)->\large
	\verb|J| operator
		{\only<.(2)->{\bf\color{highlight}}
			captures the \verb|Dump|}\cite{landin1965generalization}

	\pause
	\vspace{-1.4\zh}
	\begin{figure}[h!]
		\begin{tikzpicture}
			\node[rotate=45] (pointup) {\Large\emoji{backhand-index-pointing-right}};
			\node[below = .01 of pointup, anchor=north] {{\large\emoji{face-with-monocle} Could be seen as \textbf{\Large continuations!}\cite{10.1023/A:1010060315625}\cite{danvy2004secd}}};
		\end{tikzpicture}
	\end{figure}
\end{frame}

\subsection{\texttt{call/cc}: \texttt{goto} Practical}
\begin{frame}[fragile]
	\frametitlesubs

	In 1970s, Operators in Scheme\cite{sussman1975scheme}, notably \mintinline{scheme}|call/cc|\cite{scheme1978revised}\footnote{abbrev of "call-with-current-continuation"}:

	Captures the current continuation as a first-class value

	\begin{minted}{racket}
    (let {[(x (call/cc (λ (?\large\textcolor{highlight}{κ}?)
                (+ 2 (?\large\textcolor{highlight}{κ}? 4)))))]}
        (+ x 3))
    ; => (let {[x 4]} (+ x 3))
    ; => returns 7
  \end{minted}

	\pause
	\mintinline{scheme}{call/cc} enables to implement:

	\begin{minipage}{.3\textwidth}
		\begin{itemize}
			\item[\emoji{check-mark-button}] non-local exit
			\item[\emoji{check-mark-button}] backtracking
			\item[] $\cdots\cdots$
		\end{itemize}
	\end{minipage}
	\begin{minipage}{.64\textwidth}
		\pause
		{\Huge \emoji{backhand-index-pointing-right}}
		\Large
		\parbox{.85\textwidth}{%
			\color{highlight}
			\bf
			\centering
			ANYTHING related to

			\boldslant control transfer%
		}
	\end{minipage}
\end{frame}

\subsection{Continuation-Passing Style}
\begin{frame}[fragile]
	\frametitlesubs

	A program representation where control flow is made \textit{\semibf explicit}
	\\by chaining computations as \textcolor{subhighlight}{continuations}\cite{reynolds1972definitional}:

	\def\relanime{4}

	\begin{center}
		\only<.(\relanime)->{%
			\setminted{fontsize=\smaller[2]}\smaller[2]
			\vskip-.8\zh\relax
		}

		\tikzstyle{every picture}+=[remember picture]
		\begin{minipage}[c]{9\zw}
			\begin{minted}{racket}
        (define (add1 x)
          (+ x 1))
        (define (mul2 x)
          (* x 2))

        ?\tikzmarknode{origord2a}?(mul2 ?\tikzmarknode{origord1a}?(add1 3)?\tikzmarknode{origord1}?)?\tikzmarknode{origord2}?
      \end{minted}
		\end{minipage}
		\pause
		\begin{minipage}[c]{6\zw}
			\smaller
			\centering
			\color{highlight}\bf
			CPS\\
			$\underrightarrow{\textsf{Conversion!}}$
		\end{minipage}
		\begin{minipage}[c]{11\zw}
			\begin{minted}{racket}
        (define (add1 x ?\larger\textcolor{highlight}{κ}?)
          (?\larger\textcolor{highlight}{κ}? (+ x 1)))
        (define (mul2 x ?\larger\textcolor{highlight}{κ}?)
          (?\larger\textcolor{highlight}{κ}? (* x 2)))

        ?\tikzmarknode{cpsord1a}?(add1 3?\tikzmarknode{cpsord1}? (λ (smu)
          ?\tikzmarknode{cpsord2a}?(mul2 sum?\tikzmarknode{cpsord2}? (λ (mul)
            mul))))
      \end{minted}
		\end{minipage}
	\end{center}

	\pause{}
	CPS \underline{\textit{fixes}} the order of evaluation and control flow%
	\begin{tikzpicture}[overlay, remember picture, every node/.style={inner sep=0pt}, underline/.style={line width=1pt,draw}]
		% \smaller
		\only<.(1)->{\smaller[2]}
		\node[color=SeaGreen,circle,draw,anchor=south east,yshift=.7\baselineskip] at (origord1) {1};
		\path[underline, color=SeaGreen] ([yshift=-.2\zh]origord1a.north) -- ([yshift=-.2\zh]origord1.north);

		\node[color=Peach,circle,draw,anchor=south east,yshift=.7\baselineskip, xshift=.5\zw] at (origord2) {2};
		\path[underline, color=Peach] ([yshift=-.4\zh]origord2a.north) -- ([yshift=-.4\zh]origord2.north);

		\node[color=SeaGreen,circle,draw,anchor=south east,yshift=.7\baselineskip, xshift=.3\zw] at (cpsord1)  {1};
		\path[underline, color=SeaGreen] ([yshift=-.2\zh]cpsord1a.north) -- ([yshift=-.2\zh]cpsord1.north);

		\node[color=Peach,circle,draw,anchor=north east,yshift=-.2\baselineskip, xshift=.7\zw] at (cpsord2)  {2};
		\path[underline,color=Peach] ([yshift=-.2\zh]cpsord2a.north) -- ([yshift=-.2\zh]cpsord2.north);
	\end{tikzpicture}%
	\pause{},

	so that it's good for

	\centerline{\Large\textbf{an intermediate representation}}

	for language implementations!
\end{frame}

\subsection{Compiling with Continuations}
\begin{frame}[fragile]
	\frametitlesubs

	CPS as \textcolor{subhighlight}{an intermediate representation} (IR) for language impls\cite{appel1992cwc}:

	\begin{center}
		\begin{tikzpicture}
			\smaller[2]
			\node[draw] (src) {Source};
			\node[right = of src,draw] (cps) {\color{highlight}\larger CPS};
			\node[right = 1.8 of cps, draw] (gen) {Codegen};
			\node (opts) at ($(cps.south east)!0.5!(gen.south west)$) {(Opts.)};
			\draw[->,thick]
			(src) edge (cps)
			(cps) edge (gen);
		\end{tikzpicture}
	\end{center}

	\pause
	\begin{itemize}
		\item[\emoji{check-mark-button}]<+->
		      Good for {\only<.>{\color{subhighlight}}functional languages}

		      \smaller
		      CPS operates functions as first-class values!
		      \tikz[remember picture, overlay]\node[left = 0.2 of src.west]{
			      \scalebox{.5}{
				      $\left.
					      \begin{array}{l}
						      \textsf{SML/NJ} \\
						      \textsf{Scheme} \\
						      \begin{array}{c}
							      \textsf{OCaml} \\
							      {\scriptsize\left(\textsf{flambda}\right)}
						      \end{array}
					      \end{array}
					      \right\}$
			      }
		      };

		\item[\emoji{check-mark-button}]<+->
		      Good for \textcolor{subhighlight}{optimizations}

		      \smaller
		      Several optimizations can be done by \textcolor{highlight}{β}/\textcolor{highlight}{η},

		      and each values are single-assignment!
		      \smaller[3]
		      \begin{lrbox}{\mintedbox}
			      \begin{minipage}{14.5\zw}
				      \setminted{fontsize=\relsize{0}}
				      \begin{minted}{OCaml}
                let z = a * b + a * b
                in e
              \end{minted}
			      \end{minipage}
		      \end{lrbox}
		      \begin{tikzpicture}[remember picture, overlay]
			      \node[below right = 0.05 of opts,xshift=-2\zw] (optmap) {
				      \begin{minipage}[c]{.35\textwidth}
					      \begin{tabular}{crc}
						      \begin{tabular}{c}
							      Constant Folding \\
							      \& Inlining
						      \end{tabular}
						                          & $\Rightarrow$ & β reduction  \\
						      Defunctionalization & $\Rightarrow$ & η reduction  \\
						      \begin{tabular}{c}
							      Common Subexpression \\
							      Elimination
						      \end{tabular}
						                          & $\Rightarrow$ & \textit{EZ}:
					      \end{tabular}
				      \end{minipage}
			      };

			      \node[below = -.1 of optmap, xshift=5\zw] (raw) {\usebox{\mintedbox}};
		      \end{tikzpicture}

		      \begin{lrbox}{\mintedbox}
			      \begin{minipage}{15\zw}
				      \setminted{fontsize=\relsize{0}}
				      \begin{minted}{OCaml}
                ?\tikzmarknode{ab1}{}?( *' ) a b?\tikzmarknode{ab1e}{}? (fun ?$\texttt{v}_1$? ->
                ?\tikzmarknode{ab2}{}?( *' ) a b?\tikzmarknode{ab2e}{}? (fun ?$\texttt{v}_2$? ->
                ( +' ) ?$\texttt{v}_1$? ?$\texttt{v}_2$? (fun z ->
                ?$\llbracket\texttt{e}\rrbracket^\kappa$?
              \end{minted}
			      \end{minipage}
		      \end{lrbox}
		      \begin{tikzpicture}[remember picture, overlay]
			      \node[below = .0 of raw] (cpsed) {\usebox{\mintedbox}};
			      \path[draw] (raw.east) edge[bend left=30,->] (cpsed.east) node[midway,right]{
					      \smaller
					      \begin{tabular}{c}
						      CPS \\
						      Conversion
					      \end{tabular}
				      };

			      \path[draw, line width=0.1\zh,color=highlight]
			      ([yshift=-.2\zw]ab1.south) -- ([yshift=-.2\zw]ab1e.south)
			      ([yshift=-.2\zw]ab2.south) -- ([yshift=-.2\zw]ab2e.south);
			      \node[above= .2 of ab1e.north east,fill=white,opacity=0.5,text opacity=1] {\color{highlight}same!};
		      \end{tikzpicture}
	\end{itemize}
\end{frame}

\subsection{Delimited Coninuations}
\begin{frame}[fragile]
	\frametitlesubs

	\begin{minipage}{.09\textwidth}
		\
	\end{minipage}
	\begin{minipage}{.9\textwidth}
		\begin{itemize}
			\item[\textcolor{subhighlight}{Problem}:]
			      \mintinline{scheme}{call/cc} is a powerful control structure, but captures the entire rest of the computation (like \mintinline{c}{goto}!)

		\end{itemize}
	\end{minipage}

	\pause
	\vspace{1\zh}
	\hspace{.2\zw}\textcolor{highlight}{Solution}:

	\hspace{.3\zw}
	\begin{minipage}[t]{.60\textwidth}
		\vspace{.2\zh}
		\centering
		{\Large{}Use}

		\textbf{\large{}Delimited continuations!}\cite{felleisen1988prompt}

		\begin{itemize}
			\item[\emoji{check-mark-button}] capturing \textit{scoped} conitnuations
			\item[\emoji{check-mark-button}] more structured and composable
		\end{itemize}

	\end{minipage}
	\begin{minipage}[t]{.30\textwidth}
		\begin{figure}[h]
			\includegraphics[width=\textwidth]{img/pointing-smile-man.png}
		\end{figure}
	\end{minipage}
\end{frame}

\begin{frame}[fragile]
	\frametitlesubs

	shift/reset \cite{danvy1990abstracting}:

	captures a scoped continuation

	\begin{minted}{racket}
    (+ 3 (reset (+ 4 (shift k (k (k 5))))))

    ; => (+ 3 (+ 4 (+ 4 5)))
  \end{minted}

	\pause
	Delimited continuations enable to implement

	\begin{itemize}
		\item[\emoji{check-mark-button}] \mintinline[fontsize=\large]{scheme}|call/cc|!
		\item[\emoji{check-mark-button}] \textcolor{subhighlight}{\bf ALL of \underline{monads}!!}

		      \pause\LARGE
		\item[] $\dots\dots$and \textcolor{highlight}{\bf visa-versa}\emoji{exclamation}\cite{dybvig2005a}
	\end{itemize}
\end{frame}

\subsection{Correspondence between procedural and functional}
\begin{frame}[fragile]
	\frametitlesubs

	How do \textit{control transfer} correspond in \textcolor{subhighlight}{imperative} and \textcolor{highlight}{functional}?

	\pause
	\begin{itemize}
		\item<+-> Imperative:

		      \begin{itemize}
			      \item[\emoji{down-arrow}] \verb|jmp|: Low-level control transfer
			      \item[\emoji{down-arrow}] \verb|goto|: Arbitrary jumps in high-level repr.
			      \item[\emoji{backhand-index-pointing-right}] \verb|for| / \verb|while| / \verb|if|: Structured, clear control
		      \end{itemize}

		\item<+-> Functional:

		      \begin{itemize}
			      \item[\emoji{down-arrow}] \verb|J|: Low-level continuation operator
			      \item[\emoji{down-arrow}] \verb|call/cc|: Capturing entire continuation
			      \item[\emoji{backhand-index-pointing-right}] \verb|shift/reset|: Structured, modular and composable control
		      \end{itemize}
	\end{itemize}

	Also, CPS <-> SSA! \cite{10.1145/278283.278285}
\end{frame}


