\section{Introduction}
\begin{frame}
    \frametitlesec

    Algebraic Effectsの様々な実装

    \pause
    \begin{itemize}
        \item<+-> \itemheader{ライブラリ}
            \begin{itemize}
                \item[\hlinkclip{https://github.com/Nymphium/eff.lua}] eff.lua
                \item[\hlinkclip{https://b-studios.de/scala-effekt/}] Effekt
                \item[\hlinkclip{https://github.com/koka-lang/libhandler}] libhandler
                \item etc.
            \end{itemize}

        \item<+-> \itemheader{言語(処理系)} \hspace{11\zw}\tikz\coordinate[remember picture](think);\null
            \begin{itemize} 
                \item[\hlinkclip{https://github.com/matijapretnar/eff}] Eff
                \item[\hlinkclip{https://github.com/ocaml-multicore/ocaml-multicore}] Multicore OCaml
                \item[\hlinkclip{https://github.com/koka-lang/koka}] Koka
                \item etc.
            \end{itemize}
    \end{itemize}

    \onslide<+->{
        \begin{tikzpicture}[overlay]
            \LARGE
            \coordinate[below = 0.5 of think] (thinker0);
            \node[below = 0 of thinker0] (thinker) {\Huge \coloremoji{🤔}};

            \node[draw, rectangle callout, above left = 0.6 of thinker0, xshift = 3\zw, fill=cream, callout absolute pointer={($(thinker.north)-(.3,.1)$)}, text width=6.5\zw] (callout1) {どうやって実装\\されているの??};
        \end{tikzpicture}
    }
\end{frame}
