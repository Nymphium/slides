\section{Multiprompt shift/reset}
\begin{frame}[t,fragile]
  \frametitlesec

  \pause
  e.g.)\texttt{racket/control} in Racket

  \begin{onlyenv}<2>
    \begin{lstlisting}[language=,xleftmargin=1\zw,xrightmargin=1\zw,belowskip=-1\zh]
(reset
  (+ 2 (reset
    (+ 3 (shift _k 4)))))

;; !$\rightarrow^\star$! (reset (+ 2 4)) !$\rightarrow^\star$! 6
    \end{lstlisting}
  \end{onlyenv}
  \begin{onlyenv}<3->
    \begin{lstlisting}[language=,xleftmargin=1\zw,xrightmargin=1\zw]
(let ((p (make-continuation-prompt-tag))
      (q (make-continuation-prompt-tag)))
  (reset-at !\only<3>{\texttt{p}}\only<4>{\textcolor{red}{\textbf{p}}}\tikz\coordinate (resetp);!
    (+ 2 (reset-at !\only<3>{\texttt{q}}\only<4>{\textcolor{blue}{\textbf{q}}}!
      (+ 3 (shift-at !\only<3>{\texttt{p}}\only<4>{\textcolor{red}{\textbf{p}}}\tikz\coordinate (shiftp);! _k 4))))))

;; !$\rightarrow^\star$! (reset-at !\only<3>{\texttt{p}}\only<4>{\textcolor{red}{\textbf{p}}}! (+ 2 (shift-at !\only<3>{\texttt{p}}\only<4>{\textcolor{red}{\textbf{p}}}! _k 4)))
;; !$\rightarrow^\star$! 4
    \end{lstlisting}
  \end{onlyenv}
  \begin{onlyenv}<4>
    \begin{tikzpicture}[overlay]
      \node[above right = of resetp,draw,fill=cream,yshift=1\zh] (callout) {対応するpromptまで飛んでいく};
      \path[->,line width=2pt]
        (callout) edge ($(resetp)+(.05, .2)$)
        (callout) edge ($(shiftp)+(.1, .2)$);
    \end{tikzpicture}
  \end{onlyenv}
\end{frame}

\begin{frame}
  \frametitlesec

  e.g.) \hlinkclip{https://b-studios.de/scala-effekt/} \alert{Effekt}, implemented in \structure{Scala}

  \pause
  \begin{table}[ht]
    \centering
    \begin{tabular}{rcl}
      \alert{Algebraic Effects} & $\mapsto$ & \parbox{6em}{\structure{Multiprompt\\shift/reset}}\\
      \hline
      effect operation  & $\mapsto$ & prompt tag\\
      effect invocation & $\mapsto$ & \texttt{shift-at}\\
      effect handler    & $\mapsto$ & \texttt{reset-at}\\
      continuation & $\mapsto$ & continuation
    \end{tabular}
  \end{table}
\end{frame}

\subsection*{pros/cons}
\begin{frame}
  \frametitlesubs

  \pause
  \begin{itemize}
    \item<+->[\coloremoji{😁}] \itemheader{直感的で素直な対応}

      実装しやすい
    \item<:->[\coloremoji{😁}] \itemheader{effectのdynamic instantiationも対応}

      multi-stateなど
    \item<+->[\coloremoji{😅}] \itemheader{あまり一般的でない}

      実装が少ない
  \end{itemize}
\end{frame}
