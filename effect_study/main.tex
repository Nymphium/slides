\documentclass[unicode,compress,14pt,CJK%
\directlua{
    handout = os.getenv"HANDOUT"
    local _ = handout and tex.print(",handout")
}]{beamer}
\usepackage[no-math]{luatexja-fontspec}
\usepackage{fontawesome}
\usefonttheme[onlymath]{serif}
\usepackage{luacode}
\input{../template/beamertemp}
\usepackage{ulem}
\defaultfontfeatures{Ligatures={NoRequired, NoCommon, NoContextual}}
\defaultjfontfeatures{Ligatures={NoRequired, NoCommon, NoContextual}}
\setsansfont[Script=Default,Kerning=On,BoldFont=GenShinGothic-Bold,ItalicFont=NotoSansDisplay-LightItalic]{GenShinGothic-Light}
\setmainjfont[%
Script=Default,%
Kerning=On,%
CharacterWidth=AlternateProportional,%
BoldFont=GenShinGothic-P-Bold,%
YokoFeatures={JFM=prop}]{GenShinGothic-P-Light}
\setmonofont[Kerning=Reset]{Inconsolata}
\newfontfamily{\bfslant}[Script=Default,Kerning=On]{NotoSansDisplay-BoldItalic}
\newjfontfamily{\semijbf}[%
Script=Default,%
Kerning=On,%
CharacterWidth=AlternateProportional,%
YokoFeatures={JFM=prop}]{GenShinGothic-P-Regular}
\newfontfamily{\semibf}[%
Script=Default,%
Kerning=On,%
ItalicFont=NotoSansDisplay-Italic,
CharacterWidth=AlternateProportional,%
]{GenShinGothic-P-Regular}
\newcommand{\textbfslant}[1]{{\bfslant #1}}
\usepackage{amsmath,amssymb,stmaryrd}
\usepackage{tikz,pgfplots}
\usepackage{url,hyperref}

\usepackage[style=alphabetic,minalphanames=3,backend=biber]{biblatex}
\renewcommand*{\labelalphaothers}{\textsuperscript{+}}
\addbibresource{main.bib}
\beamertemplatetextbibitems
%% bib underline breaklines https://tex.stackexchange.com/questions/200997/underlined-titles-in-bibliography-with-biblatex-and-ulem-packages
\renewbibmacro*{title}{%
  \ifboolexpr{
    test {\iffieldundef{title}}
    and
    test {\iffieldundef{subtitle}}
  }
    {}
    {\printtext{%
     \printtext[titlecase]{\usefield{\uline}{title}}%
     \setunit{\subtitlepunct}%
     \printfield[titlecase]{subtitle}}%
     \newunit}%
  \printfield{titleaddon}}



\usepackage{graphicx}
\usepackage{xcolor}
\usepackage[most]{tcolorbox}
\newtcolorbox{newbie}{
    enhanced,
    boxrule=0pt,
    right skip=0pt,
    left skip=0pt,
    boxsep=0pt,
    right=.3em,
    left=.3em,
    sharp corners,
    colback=black!20,
}
\usetikzlibrary{shapes,snakes,arrows.meta,positioning,calc,fit,shapes.callouts}
\tikzstyle{every picture} +=[remember picture]

\usepackage{bxcoloremoji}

\def\green#1{\textcolor{green!70!black}{#1}}
\def\blue#1{\textcolor{blue!70!black}{#1}}
\def\yellow#1{\textcolor{orange!70!yellow}{#1}}

\lstset{
	basicstyle=\ttfamily\small,
	commentstyle=\color{gray},
	keywordstyle=\color{orange},
	stringstyle=\color{purple},
	showstringspaces=false,
	frame=single,
	breaklines=true,
	columns=flexible
}


%% for beamer not to underline
\def\emph#1{\textit{#1}}

\def\hlinkclip#1{%
  \href{#1}{\small\textcolor{gray}{\faicon{paperclip}}}}

\AtBeginSubsection{}

\setbeamertemplate{frametitle}{%
  \ifnoframetitle\else%
    \ifx\insertframesubtitle\empty%
      \begin{beamercolorbox}{frametitle}%
        \noindent\parbox[b][1.8\baselineskip]{\textwidth}{\centering\LARGE\semibf\semijbf\insertframetitle}\par%
        \vskip-.4\zh\xhrulefill{titleblue}{1pt}
      \end{beamercolorbox}%
    \else%
      \begin{beamercolorbox}{framesubtitle}%
        \noindent\parbox[b][1.5\baselineskip]{\textwidth}{\centering\Large\semibf\semijbf\insertframesubtitle}\par%
        \vskip-.27\zh\xhrulefill{titleblue}{1pt}\par%
        \vskip-9pt\xhrulefill{titleblue}{1pt}
      \end{beamercolorbox}
    \fi\fi}%

% \includeonly{tex/selfintro}

\definecolor{cream}{HTML}{FFFFCC}

%%%%
\title{\textcolor{black}{\Huge \textbf{How do you} \textbfslant{implement}\\\textbf{Algebraic Effects?}}}
\def\thetitle{\normalsize How do you implement Algebraic Effects?}
\author{びしょ〜じょ}
\date{May 26, 2019}
\institute{effect system勉強会}
%%%%

\begin{document}
\maketitle

\switchfooter
\begin{frame}{やること}
    \begin{center}
        \large
        \textbf{\Huge\alert{Algebraic Effects and Handlers}}

        \textbf{の}

        \textbf{\LARGE\structure{さまざまなインプリ方法}}

        \textbf{について考える。}
    \end{center}
\end{frame}
\begin{frame}{Table of Contents}
    \begin{multicols}{2}
        \tableofcontents
    \end{multicols}
\end{frame}
\switchfooter
\notoc

\section{Who talks}
\begin{frame}
    \frametitlesec

    \begin{figure}[ht]
        \begin{tikzpicture}
            \node (img) {\includegraphics[height=2\zh]{img/yh.png}};
            \node[draw, rectangle callout, right = .7 of img, callout absolute pointer={($(img.east)+(-.1,.2) $)}, yshift = 1\zh] {こんにちは、びしょ〜じょです。};
        \end{tikzpicture}
    \end{figure}

    \vskip-\baselineskip

    \begin{itemize}
        \item \itemheader{筑波大学大学院 M2}

            プログラム言語や型とか検証などの研究室で\\プログラム変換の研究

        \item[\textcolor{gray}{\faicon{paperclip}}] {\small \href{https://twitter.com/Nymphium}{\textcolor{blue!60!white}{\faicon{twitter}}} \href{https://github.com/Nymphium}{\textcolor{black}{\faicon{github}}} Nymphium}
    \end{itemize}

    \vfill
\end{frame}


\subsection{バイトコード}
\begin{frame}[fragile]
	\frametitlesubs
	\begin{itemize}
		\item \lstinline{luac}コマンドでバイトコードを生成
		\item \lstinline{luac -l bytecode.out}でいい感じに出力
		\item \lstinline{luac -l -l bytecode.out}で情報が増える
	\end{itemize}
\end{frame}
\begin{frame}[fragile]
	\frametitlesubs
ひとまず標準入力から渡してみる
	\small
	\begin{lstlisting}[numbers=none,language=sh]
cat <<LUACODE | luac -l -
print("hello, world")
LUACODE
	\end{lstlisting}
	\pause
	\begin{center}$\Downarrow$\end{center}
	\begin{lstlisting}[numbers=none]
main <stdin:0,0> (4 instructions at 0x2268790)
0+ params, 2 slots, 1 upvalue, 0 locals, 2 constants, 0 functions
        1       [1]     GETTABUP        0 0 -1  ; _ENV "print"
        2       [1]     LOADK           1 -2    ; "hello, world"
        3       [1]     CALL            0 2 1
        4       [1]     RETURN          0 1\end{lstlisting}
\end{frame}
\begin{frame}[fragile]
	\frametitlesubs
	\textrm{Lua\LaTeX}にダイレクトに書いてみる
	\bgroup
	\scriptsize
	\mylisting[language={[5.3]lua}]{codes/dumpdemo.lua}
	\npause{0}
	\tiny
	\directlua{dofile("codes/dumpdemo.lua")}
	\egroup
\end{frame}
\subsection{読み方}
\begin{frame}[fragile]
	\frametitlesubs
	\tiny
	\directlua{dofile("codes/dumpdemo.lua")}
	\normalsize

	まずこれを読む。
\end{frame}
\subsubsection{header block}
\begin{frame}[fragile]{\insertsubsubsectionhead}
	\directlua{dofile("codes/headerprint.lua")}

	header block
	\begin{multicols}{2}
		\begin{itemize}
			\item \directlua{myutil.itemprint("4bytes", 2)}

				\uncover<2->{header signature {\small{}(ESC ``Lua'')}}
			\item \directlua{myutil.itemprint("1byte")}

				\uncover<3->{Lua Version

				{\small{}(例では52、つまりLua 5.2)}}
			\item \directlua{myutil.itemprint("1byte")}

				\uncover<4->{Format Version
			
				{\small{}(0 = official(default))}}
			\item \directlua{myutil.itemprint("1byte")}

				\uncover<5->{Endianness flag

				{\small{}(1 = little endian(default), 0 = big endian)}}
			\item \directlua{myutil.itemprint("1byte")}

				\uncover<6->{size of \lstinline{int} {\small{}(C lang)}}
			\item \directlua{myutil.itemprint("1byte")}

				\uncover<7->{size of \lstinline{size_t} {\small{}(C lang)}}
			\item \directlua{myutil.itemprint("1byte")}

				\uncover<8->{size of instruction}
			\item \directlua{myutil.itemprint("1byte")}

				\uncover<9->{size of \lstinline{lua_Number}\footnote[frame]{32bit integer}{\small{}(C lang)}}
			\item \uncover<10->{\textcolor{red}{1byte}

					Integral flag

					{\small{}(0 = floating point(default), 1 = integral number type)}}
		\end{itemize}
	\end{multicols}
\end{frame}
\begin{frame}{\insertsubsubsectionhead}
	\textcolor{gray}{1B 4C 75 61 52 00 01 04 08 04 08 00} \textcolor{red}{19 93 0D 0A 1A 0A}
	\begin{itemize}
		\item 6bytes

		\lstinline{LUAC_DATA}, ``data to catch conversion errors''(ソースのコメントより)

		\begin{itemize}
			\item 19 93

				Lua1.0のリリース年
			\item 0D 0A

				DOSの改行(CR LF)、DOS$\rightarrow$UNIXでの改行のデータ変換の検知
			\item 1A

				SUB
			\item 0A

				UNIXの改行(LF)、UNIX$\rightarrow$DOSでの改行のデータ変換の検知
		\end{itemize}
\end{itemize}
\end{frame}
\begin{frame}{\insertsubsubsectionhead}
	ちなみに

	Lua5.2

	\begin{center}
		1B 4C 75 61 52 00 01 04 08 04 08 00 19 93 0D 0A 1A 0A

		$\Downarrow$
	\end{center}

	Lua5.3

	\begin{center}
		\ltjruby{\textcolor{red}{1B 4C 75 61}}{header signature} \ltjruby{\textcolor{blue}{53}}{version}
		\ltjruby{\textcolor{red}{00}}{format version} \ltjruby{\textcolor{blue}{19 93 0D 0A 1A 0A}}{LUAC\_DATA}
		\ltjruby{\textcolor{red}{04 08 04 08 08}}{objects sizes} \ltjruby{\textcolor{blue}{78 56 00 00 00 00 00 00}}{LUAC\_INT\footnote[frame]{Endiannessの検査、0x5678をdumpしているのでこの場合リトルエンディアン}}
		\ \ltjruby{\textcolor{red}{00 00 00 00 00 00 00 28 77 40}}{LUAC\_NUM\footnote[frame]{IEEE754 float formatの検査}}
	\pause

	\alert{\Huge{}長い}
	\end{center}
\end{frame}
\subsubsection{function block}
\begin{frame}[fragile]{\insertsubsubsectionhead}
	02 00 00 00 02 00 00 00 00 00 02 01 00$\cdots\cdots$

	function block
	\scriptsize
	\begin{multicols}{2}
		\begin{itemize}
			\item \lstinline{int}

				line defined
			\item \lstinline{int}

				last line defined
			\item 1byte

				number of parameters
			\item 1byte

				\lstinline{is_vararg}
			\item 1byte

				number of registers used by the function
			\item List

				\textcolor<2->{red}{number/list of instructions}
			\item List

				\textcolor<2->{red}{number/list of constants}
			\item List

				\textcolor<2->{red}{number/list of upvalues}
			\item List

				debug info
		\end{itemize}
	\end{multicols}
\end{frame}

\section{Low-level Manipulations}
\begin{frame}
  \frametitlesec

  e.g.) \hlinkclip{https://github.com/koka-lang/libhandler} \alert{libhandler}, implemented in \structure{C}

  \pause
  \begin{table}[ht]
    \centering

    \begin{tabular}{ccc}
      \alert{Algebraic Effects} & $\mapsto$ & \structure{libhandler} \\\hline
      \parbox{8em}{\centering effect handler \& \\ ciontinuation} & $\mapsto$ & \parbox{9em}{\centering call stack \& \\instruction pointer}\\
      effect invocation & $\mapsto$ & longjmp
    \end{tabular}
  \end{table}
\end{frame}

\subsection*{pros/cons}
\begin{frame}
  \frametitlesubs

  \pause
  \begin{itemize}
    \item<+->[\coloremoji{😁}] 低レベルで提供可能

      FFIで様々な言語から呼び出せる

    \item<+->[\coloremoji{😓}] 実装が大変かつ限定的
  \end{itemize}

  \onslide<+->{
    \centering
    $\Downarrow$

    \{処理系, ライブラリ\}バックエンド向け
  }
\end{frame}


\section{Coroutines}
\begin{frame}
  \frametitlesec

  \begin{center}
    \begin{minipage}[c]{1.8em}
      e.g.)
    \end{minipage}
    \begin{minipage}[c]{.8\textwidth}
      \begin{itemize}
        \item[\hlinkclip{https://github.com/ocaml-multicore/ocaml-multicore}] \alert{Multicore OCaml}, implemented\\with \structure{fiber}, in \structure{C}~\cite{dolan2015effective} 
        \item[\hlinkclip{https://github.com/Nymphium/eff.lua}] \alert{eff.lua} , implemented with \structure{coroutine}, in \structure{Lua}
      \end{itemize}
    \end{minipage}
  \end{center}

  \pause

  \begin{table}[ht]
    \centering

    \setbeamercovered{transparent}

    \begin{tabular}{rcl}
      \alert{Algebraic Effects} & $\mapsto$ & \structure{Coroutines}\\\hline
      effect invocation & $\mapsto$ & yield\\
      effect handler & $\mapsto$ & create \& resume\\
      continuation & $\mapsto$ & coroutine
    \end{tabular}
  \end{table}
\end{frame}

\begin{frame}
  \frametitlesec

  \begin{table}[ht]
    \centering

    \setbeamercovered{transparent}

    \begin{tabular}{rcl}
      \uncover<1>{\alert{Algebraic Effects} & $\mapsto$ & \structure{Coroutines}\\\hline}
      \uncover<1>{effect invocation & $\mapsto$ & yield}\\
      \uncover<1>{effect handler & $\mapsto$ & create \& resume}\\
      \uncover<1->{\tikz\coordinate[remember picture] (continuation);\null continuation & $\mapsto$ & coroutine\tikz\coordinate[remember picture] (coroutine);\null }
    \end{tabular}
  \end{table}

  \begin{onlyenv}<3->
    \begin{tikzpicture}[overlay]
      \node[
        draw,
        rectangle callout,
        fill=cream,
        above = of continuation,
        xshift=.5\zw,
        callout absolute pointer={($(continuation) + (1, .4) $)}]
        {何回も呼び出したいけど};
      \node[
        draw,
        rectangle callout,
        fill=cream,
        above=of coroutine,
        callout absolute pointer={($(coroutine) - (1, -.4)$)}
        ]
        {状態をコピーできない};
    \end{tikzpicture}
  \end{onlyenv}

  \onslide<4->{
    \centering
    \begin{minipage}{1em}
      \Large \coloremoji{⚠}%
    \end{minipage}
    \begin{minipage}{.9\textwidth}
      \centering coroutinesをコピーするような操作がなければ\\
      継続は\alert{ワンショットに限定}される
    \end{minipage}
  }
\end{frame}

\subsection*{pros/cons}
\begin{frame}[fragile]
  \frametitlesubs

  \pause
  \begin{itemize}
    \item<+->[\coloremoji{😁}] \itemheader{さまざまな言語で実装可能}

      Coroutinesを持ってる言語は多い\coloremoji{😉}

      {\small Lua, Ruby, JS, Kotlin, Python, etc.}

    \item<+->[\coloremoji{😅}] \itemheader{継続はワンショット}

      非決定計算とかは書けない

    \item<+->[\coloremoji{🤔}] \itemheader{coroutineを複製する操作があれば……}

      \alert{Multicore OCaml}の\lstinline{Obj.clone_continuation}の実装
  \end{itemize}
\end{frame}

\section{Multiprompt shift/reset}
\begin{frame}[t,fragile]
  \frametitlesec

  \pause
  e.g.)\texttt{racket/control} in Racket

  \begin{onlyenv}<2>
    \begin{lstlisting}[language=,xleftmargin=1\zw,xrightmargin=1\zw,belowskip=-1\zh]
(reset
  (+ 2 (reset
    (+ 3 (shift _k 4)))))

;; !$\rightarrow^\star$! (reset (+ 2 4)) !$\rightarrow^\star$! 6
    \end{lstlisting}
  \end{onlyenv}
  \begin{onlyenv}<3->
    \begin{lstlisting}[language=,xleftmargin=1\zw,xrightmargin=1\zw]
(let ((p (make-continuation-prompt-tag))
      (q (make-continuation-prompt-tag)))
  (reset-at !\only<3>{\texttt{p}}\only<4>{\textcolor{red}{\textbf{p}}}\tikz\coordinate (resetp);!
    (+ 2 (reset-at !\only<3>{\texttt{q}}\only<4>{\textcolor{blue}{\textbf{q}}}!
      (+ 3 (shift-at !\only<3>{\texttt{p}}\only<4>{\textcolor{red}{\textbf{p}}}\tikz\coordinate (shiftp);! _k 4))))))

;; !$\rightarrow^\star$! (reset-at !\only<3>{\texttt{p}}\only<4>{\textcolor{red}{\textbf{p}}}! (+ 2 (shift-at !\only<3>{\texttt{p}}\only<4>{\textcolor{red}{\textbf{p}}}! _k 4)))
;; !$\rightarrow^\star$! 4
    \end{lstlisting}
  \end{onlyenv}
  \begin{onlyenv}<4>
    \begin{tikzpicture}[overlay]
      \node[above right = of resetp,draw,fill=cream,yshift=1\zh] (callout) {対応するpromptまで飛んでいく};
      \path[->,line width=2pt]
        (callout) edge ($(resetp)+(.05, .2)$)
        (callout) edge ($(shiftp)+(.1, .2)$);
    \end{tikzpicture}
  \end{onlyenv}
\end{frame}

\begin{frame}
  \frametitlesec

  e.g.) \hlinkclip{https://b-studios.de/scala-effekt/} \alert{Effekt}, implemented in \structure{Scala}

  \pause
  \begin{table}[ht]
    \centering
    \begin{tabular}{rcl}
      \alert{Algebraic Effects} & $\mapsto$ & \parbox{6em}{\structure{Multiprompt\\shift/reset}}\\
      \hline
      effect operation  & $\mapsto$ & prompt tag\\
      effect invocation & $\mapsto$ & \texttt{shift-at}\\
      effect handler    & $\mapsto$ & \texttt{reset-at}\\
      continuation & $\mapsto$ & continuation
    \end{tabular}
  \end{table}
\end{frame}

\subsection*{pros/cons}
\begin{frame}
  \frametitlesubs

  \pause
  \begin{itemize}
    \item<+->[\coloremoji{😁}] \itemheader{直感的で素直な対応}

      実装しやすい
    \item<:->[\coloremoji{😁}] \itemheader{effectのdynamic instantiationも対応}

      multi-stateなど
    \item<+->[\coloremoji{😅}] \itemheader{あまり一般的でない}

      実装が少ない
  \end{itemize}
\end{frame}

\section{Free Monad}
\begin{frame}[t,fragile]
  \frametitlesec

  % e.g.) \hlinkclip{https://github.com/matijapretnar/eff} \alert{Eff}, implemented in \structure{\only<1-2>{OCaml}\only<3>{\LARGE OCaml}}

  \begin{onlyenv}<2>
    \begin{lstlisting}[language={[Objective]caml}]
type 'a free =
  | Pure: 'a !$\rightarrow$! 'a free
  | Impure: 'arg * ('res !$\rightarrow$! 'a free) !$\rightarrow$! 'a free

let rec (!$\gg=$!) op f =
  match op with
  | Pure x !$\rightarrow$! f x
  | Impure (x, k) !$\rightarrow$!
    Impure (x, fun y !$\rightarrow$! k y !$\gg=$! f)
    \end{lstlisting}
  \end{onlyenv}
  % \begin{onlyenv}<3>
    % \begin{lstlisting}[language={[Objective]caml}]
% type ('arg, 'res) operation = ..

% (* effect Decide : unit !$\rightarrow$! bool *)
% type (_, _) operation +=
  % | Decide : (unit, bool) operation

% type 'a free =
  % | Pure: 'a !$\rightarrow$! 'a free
  % | Impure:   ('arg, 'res) operation * 'arg
            % * ('res !$\rightarrow$! 'a free) !$\rightarrow$! 'a free
    % \end{lstlisting}
  % \end{onlyenv}
\end{frame}

\begin{frame}
  \frametitlesec

  e.g.) \hlinkclip{https://github.com/matijapretnar/eff} \alert{Eff}, implemented in \structure{OCaml}

  \pause
  \begin{table}[ht]
    \centering
    \begin{tabular}{ccc}
      \alert{Algebraic Effects}   & $\mapsto$ & \structure{Free Monad}\\\hline
      effect invocation   & $\mapsto$ & Impure \\
      % pure computation    & $\mapsto$ & Pure\\
      % 質問で答える
      effect handler      & $\mapsto$ & run\\
      continuation        & $\mapsto$ & rhs of ($\gg=$)
    \end{tabular}
  \end{table}
\end{frame}

\subsection*{pros}
\begin{frame}
  \frametitlesubs

  \begin{itemize}
    \item[\coloremoji{😁}] \itemheader{Freeの資産が使える}

      \cite{pretnar2017efficient}では\green{equation rules}や\green{type-directed optimisation}などを駆使して実行効率の良いコードを生成
  \end{itemize}
\end{frame}

\subsection*{cons}
\begin{frame}[fragile]
  \frametitlesubs

  \begin{itemize}
    \item[\coloremoji{😅}] \itemheader{ただのFree Monad}

      monadicな書き方ができないとちょっとつらい

      {\small Haskellの\lstinline{do}, F\#のcomputation expression,  Scalaの\lstinline{for}, etc.}
  \end{itemize}
\end{frame}

\section{\textit{N-Barrelled} CPS}
\begin{frame}[t]
  \frametitlesec

  \vskip.7\baselineskip

  \structure<1-2>{Double-Barrelled} CPS~\cite{thielecke2002comparing} を拡張するといい感じに使えるのでは?

  \pause
  \begin{onlyenv}<2-3>
    \begin{table}[ht]
      \centering
      \begin{tabular}{c|c}
        sort & \parbox{9em}{\centering number of\\continuations}\\
        \hline
        (pure) CPS & 1 \\
        \hline
        \only<2>{\coloremoji{👉}} CPS \structure<2>{+ Exception} & \structure<2>{2} \onslide<3->{\\\hline %
        CPS \alert{+ Algebraic Effects} & 1 \alert{+ \parbox{7.2em}{\centering number of\\effect handlers}}}
      \end{tabular}
    \end{table}
  \end{onlyenv}

  \begin{onlyenv}<4->
    \begin{table}[ht]
      \centering
      \begin{tabular}{ccc}
        \alert{Algebraic Effects} & $\mapsto$ & \structure{N-Barrelled CPS} \\\hline
        effect handler    & $\mapsto$ & \only<4>{(effect-id * handler) list}\only<5->{\textcolor{red}{\uwave{(effect-id * handler) list}}}\tikz[remember picture]\coordinate(thisiscont);\null\\
        effect invocation & $\mapsto$ & \parbox{8em}{\centering lookup list \&\\pass parameters}\\
        handler nesting   & $\mapsto$ & handler-list concatenation\\
      \end{tabular}
    \end{table}
  \end{onlyenv}

  \begin{onlyenv}<5>
    \begin{tikzpicture}[remember picture, overlay]
      \node[draw, rectangle callout , callout absolute pointer={($(thisiscont)+(-1,.4)$)}, above = of thisiscont,fill=cream] {N-Barrel};
    \end{tikzpicture}
  \end{onlyenv}
\end{frame}

\begin{frame}[fragile]
  \frametitlesec

  \begin{lstlisting}[title={補助関数}]
let lookup : ('a * 'b) list !$\rightarrow$! 'a !$\rightarrow$! 'b option
= !$\cdots\cdots$!
let lookup!$_\mathit{eff}$! h eff v k =
  match lookup h eff with
  | Some effh !$\rightarrow$! effh v k
  | None !$\rightarrow$! !$\mathit{abort}$!
  \end{lstlisting}
\end{frame}

\begin{frame}[t,fragile]
  \frametitlesec

  \vskip-1.5\baselineskip

  \begin{lstlisting}[title={例}]
handle (perform (Foo 5)) with
| effect (Foo x) k !$\rightarrow$! k (x + x)
| effect (Bar b) k !$\rightarrow$! k (b - b)
| (*  value   *) y !$\rightarrow$! y * 20
  \end{lstlisting}

  \pause
  \begin{center}\vskip-1em$\Downarrow$ 雰囲気で変換 \hlinkclip{https://gist.github.com/Nymphium/bb8235134121c57d8c470ee39b74f586}  \end{center}
  \vskip-1em
  \begin{lstlisting}
(fun k!$_0$! h!$_0$! !$\rightarrow$!
  (fun k!$_1$! h!$_1$! !$\rightarrow$! k!$_1$! 5) (fun v!$_1$! !$\rightarrow$!
  (lookup!$_\mathit{eff}$! h!$_0$! Foo v!$_1$!) (fun res!$_\mathtt{Foo}$! !$\rightarrow$!
  (fun y k!$_\mathit{top}$! !$\rightarrow$! k!$_\mathit{top}$! (y * 20)) res!$_\mathtt{Foo}$! k!$_0$!)) h!$_0$!
) (fun x !$\rightarrow$! x)
  [ (Foo, fun x k !$\rightarrow$! k (x + x));
    (Bar, fun b k !$\rightarrow$! k (b - b)) ]
  \end{lstlisting}
\end{frame}

\subsection*{pros/cons}
\begin{frame}
  \frametitlesubs

  \pause
  \begin{itemize}
    \item<+->[\coloremoji{😁}] CPSのテク(最適化など)の恩恵が得られそう
    \item<+->[\coloremoji{😁}] 捕捉可能なハンドラに一発で飛べる
    \item<+->[\coloremoji{😕}] グローバルな変換が必要
  \end{itemize}

  \onslide<+->{
    \begin{center}
      \centering
      $\Downarrow$

      \{処理系, ライブラリ\}バックエンド向け
    \end{center}
  }

  \begin{itemize}
    \item<+->[\coloremoji{⭐}] related) \hlinkclip{https://koka-lang/koka} Koka

      compililing to JS or C\# via type-directed \structure{\textit{selective} CPS}~\cite{leijen2016algebraic}
  \end{itemize}
\end{frame}

\section{Good Point}
\begin{frame}
  \frametitlesec

  \pause
  様々なコントロール抽象がAlgebraic Effectsで実装されていると
  ユーザの学習コストを抑えられる

  \vskip-.5\baselineskip

  \begin{figure}[ht]
    \centering
    \begin{tikzpicture}
      \onslide<3->{
        \node (thinker) {\LARGE \coloremoji{🤷‍♂️}};
        \bgroup
          \small
          \node[fill=white,draw, cloud callout, callout pointer segments = 1, cloud ignores aspect,cloud puffs=11.7, cloud puff arc=140, callout absolute pointer={(thinker.north west)} , above left = .3 of thinker,yshift=-.5\zh] (coroutines) { \structure{Coroutines} };
          \node[fill=white,draw, cloud callout, callout pointer segments = 2, cloud ignores aspect,cloud puffs=11.7, cloud puff arc=140, callout absolute pointer={(thinker.north)} , above = .8 of thinker,text width=7\zw, xshift=-4\zw] (delimcc) {\centering \structure{Delimited\\Continuations}};
          \node[fill=white,draw, cloud callout, callout pointer segments = 2, cloud ignores aspect,cloud puffs=11.7, cloud puff arc=140, callout absolute pointer={(thinker.north east)} , above right = .9 of thinker, xshift=-2.8\zw,yshift=.52\zh] (free) {\structure{Free Monad}};
          \node[fill=white,draw, cloud callout, callout pointer segments = 1, cloud ignores aspect,cloud puffs=6.3, cloud puff arc=140, callout absolute pointer={($(thinker.east)+(0,.5)$)} , right = .3 of thinker, yshift=2.4\zh] (que) {\coloremoji{😨}??};
        \egroup
      }

      \onslide<4->{
        \node[right = 2.5 of thinker,yshift=.7\zh] (okman) {\Huge \coloremoji{🙆‍♂️}};
        \coordinate[right = 9 of okman];
        \node[fill=white,draw, ellipse callout, callout absolute pointer={(okman.north west)} , above = .7 of thinker,text width=8\zw, xshift=2em] (ae) {\LARGE\color{red}Algebraic\\\ \ \,Effects};

        \path[->,line width=3pt] (thinker) edge (okman);
      }
    \end{tikzpicture}
  \end{figure}
\end{frame}

\section{まとめ}
\begin{frame}
    \frametitlesec

    \begin{itemize}
        \item[\coloremoji{🌋}] Algebraic Effectsが楽しい

            ICFP 2018やML Workshop2018にも\\
            AE関連のトピック%
            \footnote{\url{https://icfp18.sigplan.org/event/mlfamilyworkshop-2018-papers-programming-with-abstract-algebraic-effects}}%
            \footnote{\url{https://icfp18.sigplan.org/event/icfp-2018-papers-versatile-event-correlation-with-algebraic-effects}}

        \item[\coloremoji{🉐}] Algebraic Effects使おう

            \begin{itemize}
                \item[\coloremoji{🛠}] インプリいろいろ
                \item[\coloremoji{💪}] なければ自作も可
            \end{itemize}

        \item[\coloremoji{👨‍💻}] 研究やってます
    \end{itemize}
\end{frame}



\switchfooter
\begin{frame}[t]{参考文献}
    \printbibliography
\end{frame}
\end{document}
