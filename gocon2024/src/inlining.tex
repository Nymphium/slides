\section{Function Inlining}
\begin{frame}[t,fragile,containsverbatim]
	\frametitlesec

	\textcolor{highlight}{\textbf{Function inlining}}, or just \textit{inlining}, is\\ the process of \textcolor{subhighlight}{replacing a func call with its body}.

	\pause
	\centering
	\tikzstyle{every picture}+=[remember picture]
	\begin{minipage}[t]{.47\textwidth}
		\begin{onlyenv}<+->
			\begin{minted}[fontsize=\footnotesize]{go}
        func ?\tikz\coordinate(f);?f (x int) int {
          return x * x
        }

        func main() {
          fmt.printf("%d", ?\tikz\coordinate(call);?f(3))
        }
      \end{minted}
		\end{onlyenv}
	\end{minipage}
	\begin{minipage}[t]{.49\textwidth}%
		\vskip-.1\zh
		\begin{uncoverenv}<.(2)->
			\begin{minted}[fontsize=\footnotesize]{go}
        func f(x int) int {
          return x * x
        }

        func main() {
          fmt.printf("%d", ?\tikz\coordinate(inl);\colorbox{orange}{\textcolor{white}{3 * 3}}?)
        }
      \end{minted}
		\end{uncoverenv}
	\end{minipage}

	\begin{tikzpicture}[overlay, every node/.style={inner sep=0pt, outer sep=0pt}]
		\node<+-> [text width=.37\textwidth,text height=2.8\zh,fill=orange,opacity=0.4,anchor=north west] (fentire) at ([xshift=-2.2\zw,yshift=.7\zh] f) {};
		\draw<.-> [boldarrow, draw=blue, bend left=15,opacity=0.5] ([yshift=.5\zh]call) to node[above right, text opacity=1]{\footnotesize \textcolor{highlight}{see definition}} ([yshift=0pt]fentire);

		\draw<+->[boldarrow, draw=orange, bend left=10, opacity=0.6] ([xshift=.2\zw]fentire.east) to node[above right, text opacity=1]{\footnotesize \textcolor{highlight}{inline!}} ([yshift=.5\zh]inl);
	\end{tikzpicture}

	\onslide<+->
	\footnotesize
	\emoji{warning} Note: actually, performed on the IR or more lower level
\end{frame}

\begin{frame}[t]
	\frametitlesec

	Inlining has \textcolor<.-4>{subhighlight}{several benefits}:

	\onslide<+->
	\begin{itemize}[<+->]
		\item[\emoji{thumbs-up}] \textbf<.>{\only<.>{\color{subhighlight}} Reduces \textit{function call overhead}}:

		      \only<.(1)->{\footnotesize}
		      No stack frame setup, no return address, no arguments copying

		\item[\emoji{thumbs-up}] \textbf<.>{\only<.>{\color{subhighlight}}Improves \textit{cache locality}}:

		      \only<.(1)->{\footnotesize}
		      The inlined code is closer to the caller

		\item[\emoji{thumbs-up}] \textbf<.>{\only<.>{\color{subhighlight}}Enables \textit{further optimizations}}:

		      \only<.(1)->{\footnotesize}
		      E.g., \textit<.>{constant propagation}, \textit<.>{dead code elimination}
	\end{itemize}

	\onslide<+->
	\vskip-1.4\zh
	\begin{center}
		\begin{minipage}{.35\textwidth}
			\huge\bf\textcolor{subhighlight}{\scalefont{1.4} BUT$\cdots$}
		\end{minipage}
		\begin{minipage}{.6\textwidth}
			\Large
			\onslide<+->
			There are

			\textcolor{highlight}{\textbf{several conditions}}

			to be applied.
		\end{minipage}
	\end{center}
\end{frame}

\subsection{Conditions for Inlining}
\begin{frame}[t,fragile]
	\frametitlesubs

	\textcolor{highlight}{\textbf{Several conditions}} to be applied:

	\onslide<+->
	\begin{itemize}[<+->]
		% \item[\emoji{check-mark-button}] Specific tags not set,

		% {\small \texttt{//go:noinline}, \texttt{//go:systemstack}, etc.}

		\item[\emoji{check-mark-button}] {\only<.-.(2)>{\color{subhighlight}} \textbf<.-.(2)>{Non-leaf} function:}

		      \onslide<+->
		      \only<.(2)->{\footnotesize}
		      The func \textcolor<.-.(1)>{highlight}{shouldn't} call other funcs.

		      \begin{onlyenv}<.-.(1)>
			      \tikzstyle{every picture}+=[remember picture]
			      \centering
			      \begin{minipage}[t]{.5\textwidth}
				      \begin{minted}[fontsize=\footnotesize]{go}
func f (x int) int {
  return x * x
}

func g (x int) int {
  return f(x)?\tikz\coordinate[anchor=center](g);? + 1
}
              \end{minted}
			      \end{minipage}

			      \onslide<+>
			      \tikz[overlay]{
				      \node[fill=orange,anchor=center,inner sep=2pt, outer sep=2pt] at ([xshift=-.9\zw,yshift=.23\zh]g) {\footnotesize\textcolor{white}{\texttt{f(x)}}};
				      \node[mycallout, callout absolute pointer={($(g)-(.7,.1)$)}, yshift=-2.4\zh,xshift=-5\zw] (c1) at (g) {\textcolor{highlight}{calls \texttt{f}}};
				      \node[draw, xshift=7\zw] (c2) at (c1.east) {\textcolor{highlight}{\textbf{non-leaf function}}};
				      \draw[boldarrow, draw=orange] ([xshift=2pt]c1.east) to ([xshift=-2pt]c2.west);
			      }
		      \end{onlyenv}

		\item[\emoji{check-mark-button}] {\only<.>{\color{subhighlight}}\textbf<.>{Small function}, \textit<.>{"Budget"} $\leq 80$:}

		      \only<+->{
			      \setbeamertemplate{itemize/enumerate subbody begin}{\scriptsize}
			      \footnotesize
		      }

		      Constructs are \textcolor<.(-1)>{subhighlight}{rated} by their cost:\footnote{\url{https://github.com/golang/go/blob/go1.21.0/src/cmd/compile/internal/inline/inl.go}}

		      \begin{itemize}[<1->]
			      \item \textcolor<.(-1)>{subhighlight}{57} for non-leaf func call
			      \item \textcolor<.(-1)>{subhighlight}{1} for \mintinline{go}{panic}
			      \item etc.
		      \end{itemize}

		      The \textit<.(-1)>{budget} is the total cost of func body.

		\item<.-> Other conds$\cdots$
		      \begin{itemize}[<1->]
			      \item Not external function (e.g., C functions),
			      \item No specific tags set,
			      \item Not a complex body, including \mintinline{go}{defer}, \mintinline{go}{recover}, etc.
		      \end{itemize}
	\end{itemize}
\end{frame}

\subsection{Compiler Flags for Inlining}
\begin{frame}[t,fragile]
	\frametitlesubs

	Use \mintinline{sh}{gcflags} to \texttt{go build} to control compiler flags.



	\begin{table}
		\centering
		\begin{tabular}{l|l}
			\hline
			{\semibf Flag} & {\semibf Description} \\
			\hline
			\texttt{-m}    & Verbose optimization  \\
			\hline
			\texttt{-l}    & Control inlining      \\
			\hline
		\end{tabular}
	\end{table}



	% \begin{minted}{go}
	% type Point struct{ x, y int }

	% func (p Point) String() string {
	% return fmt.Sprintf("(%d, %d)", p.x, p.y)
	% }

	% func main() {
	% p := Point{1, 2}
	% fmt.Println(p)
	% }
	% \end{minted}
\end{frame}
