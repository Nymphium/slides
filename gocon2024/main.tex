\documentclass[unicode,compress,14pt,CJK,aspectratio=43,xcolor={dvipsnames}%
\directlua{
    handout = os.getenv"HANDOUT"
    local _ = handout and tex.print(",handout")
}]{beamer}
\usepackage{luatexja}
\usepackage[no-math]{luatexja-fontspec}
\usepackage{fontawesome}
\usepackage{emoji}
\usefonttheme[onlymath]{serif}
\usepackage{luacode}
\input{../template/beamertemp}
\usepackage{ulem}
% \setsansfont[Script=Default,Kerning=On,AutoFakeBold,AutoFakeSlant,BoldSlantedFeatures={AutoFakeBold,AutoFakeSlant}]{GenShinGothic-Medium}
\setsansfont[Script=Default,Kerning=On,BoldFont={*-Bold},ItalicFont=NotoSansDisplay-Italic]{GenShinGothic}
\setmainjfont[%
Script=Default,%
Kerning=On,%
YokoFeatures={JFM=prop}]{GenShinGothic-P}
\setmonofont[Kerning=Reset]{MonaspiceArNerdFont}
\newjfontfamily{\semijbf}[%
Script=Default,%
Kerning=On,%
AutoFakeSlant,BoldSlantedFeatures={AutoFakeBold,AutoFakeSlant},
YokoFeatures={JFM=prop}]{GenShinGothic-P-Medium}
\newfontfamily{\semibf}[%
Script=Default,%
Kerning=On,%
AutoFakeSlant,
BoldSlantedFeatures={AutoFakeBold,AutoFakeSlant},
% ItalicFont=NotoSansDisplay-Italic,
% CharacterWidth=AlternateProportional,%
]{GenShinGothic-P-Medium}
\usepackage{amsmath,amssymb,stmaryrd}
\usepackage{tikz,pgfplots}
\pgfplotsset{compat=1.18}
\usepackage{url,hyperref}
\usepackage{scalefnt}

\definecolor{accent}{HTML}{2978D6}
\definecolor{basetext}{RGB}{105,105,105}
\definecolor{highlight}{HTML}{D63129}
\definecolor{subhighlight}{HTML}{2978D6}

\setbeamercolor*{normal text}{fg=basetext}
\setbeamercolor*{section in head/foot}{fg=accent}%
\setbeamercolor*{subsection in head/foot}{fg=accent}%
\setbeamercolor*{frametitle}{fg=accent}%
\setbeamercolor*{framesubtitle}{fg=subhighlight}
\setbeamercolor*{headline}{fg=accent}%
\usebeamercolor[fg]{normal text}

\usepackage[newfloat=true]{minted}
\usepackage{mdframed}
\usemintedstyle{friendly}
\setminted{%
  fontsize=\footnotesize,
  beameroverlays=true,
  frame=single,
  framerule=1.3pt,
  framesep=.6\fboxsep,
  escapeinside=??,
autogobble}
% \BeforeBeginEnvironment{minted}{
% \begingroup
% \catcode`\!=\active
% \def!#1!{\colorbox{green}{#1}}}
% \AfterEndEnvironment{minted}{\endgroup}

\usepackage{graphicx}
\usepackage[most]{tcolorbox}
\newtcolorbox{newbie}{
    enhanced,
    boxrule=0pt,
    right skip=0pt,
    left skip=0pt,
    boxsep=0pt,
    right=.3em,
    left=.3em,
    sharp corners,
    colback=black!20,
}
\usetikzlibrary{tikzmark,shapes,arrows,arrows.meta,positioning,calc,fit,shapes.callouts}
% \tikzstyle{every picture}+=[remember picture]

\tikzset{
  orange arrow/.style={
    draw=orange,
    arrows={-Triangle[angle=90:10pt]},
    line width=.3\zh
    % postaction={draw=orange,shorten >=.2\zh, -}
  }
}

\newcommand{\itemheader}[1]{%
  {\semibf\semijbf #1}}

\AtBeginSection{}
\AtBeginSubsection{}

\title{%
{\huge\it\bf\textcolor{subhighlight}{Guide}}%
\newline\null%
{\large \textcolor{basetext}{to}}%
\newline\null%
{\huge\it\bf\textcolor{highlight}{ Profile-Guided Optimization:}}}
\subtitle{\bf\large\color{basetext} inlining, devirtualizing, and profiling}
\def\thetitle{\normalsize Guide to Profile-Guided Optimization: inlining, devirtualizing, and profiling}
\hypersetup{pdftitle={\thetitle}}
\author{Satoru Kawahara}
\date{June 8, 2024}
\institute{Go Conference 2024}
\begin{document}
\maketitle
\section{Who talks}

\begin{frame}
  \frametitlesec

  \begin{itemize}
    \item[\emoji{office}]
      \itemheader{\href{https://corp.eiicon.net}{eiicon, co.,ltd.}}
    \item[\textcolor{gray}{\faicon{paperclip}}] {\small
        \href{https://twitter.com/Nymphium}{\textcolor{blue!60!white}{\faicon{twitter}}}
        \href{https://github.com/Nymphium}{\textcolor{black}{\faicon{github}}}
      Nymphium}\tikzmark{item1}
    \item[\emoji{heart}] FP Lover
    \item[\emoji{nerd-face}] Interested in:
      {\small %
        \begin{itemize}
          \item Programming language theory and implementations
          \item Control-flow and its operators
        \end{itemize}}

    \item[\emoji{flexed-biceps}]<2-> Motto:

      \begin{center}
        \hspace{-2\zw}%
        \Large\boldslant Continuation is power
      \end{center}

  \end{itemize}

  \begin{tikzpicture}[remember picture, overlay]
    \node[anchor = west] (img) at ([xshift=1.05\zw,yshift=-.7\zh]pic cs:item1)
    {\includegraphics[height=4\zh]{img/yh.png}};
    \node[draw,fill=white, ellipse callout, right = .6 of img, callout
      absolute pointer={($(img.east)+(-.2,.2) $)}, xshift=-1\zw,
    yshift=2\zh, align=center] {こんにちは、\\びしょ〜じょです。};
  \end{tikzpicture}
\end{frame}


\subsection{バイトコード}
\begin{frame}[fragile]
	\frametitlesubs
	\begin{itemize}
		\item \lstinline{luac}コマンドでバイトコードを生成
		\item \lstinline{luac -l bytecode.out}でいい感じに出力
		\item \lstinline{luac -l -l bytecode.out}で情報が増える
	\end{itemize}
\end{frame}
\begin{frame}[fragile]
	\frametitlesubs
ひとまず標準入力から渡してみる
	\small
	\begin{lstlisting}[numbers=none,language=sh]
cat <<LUACODE | luac -l -
print("hello, world")
LUACODE
	\end{lstlisting}
	\pause
	\begin{center}$\Downarrow$\end{center}
	\begin{lstlisting}[numbers=none]
main <stdin:0,0> (4 instructions at 0x2268790)
0+ params, 2 slots, 1 upvalue, 0 locals, 2 constants, 0 functions
        1       [1]     GETTABUP        0 0 -1  ; _ENV "print"
        2       [1]     LOADK           1 -2    ; "hello, world"
        3       [1]     CALL            0 2 1
        4       [1]     RETURN          0 1\end{lstlisting}
\end{frame}
\begin{frame}[fragile]
	\frametitlesubs
	\textrm{Lua\LaTeX}にダイレクトに書いてみる
	\bgroup
	\scriptsize
	\mylisting[language={[5.3]lua}]{codes/dumpdemo.lua}
	\npause{0}
	\tiny
	\directlua{dofile("codes/dumpdemo.lua")}
	\egroup
\end{frame}
\subsection{読み方}
\begin{frame}[fragile]
	\frametitlesubs
	\tiny
	\directlua{dofile("codes/dumpdemo.lua")}
	\normalsize

	まずこれを読む。
\end{frame}
\subsubsection{header block}
\begin{frame}[fragile]{\insertsubsubsectionhead}
	\directlua{dofile("codes/headerprint.lua")}

	header block
	\begin{multicols}{2}
		\begin{itemize}
			\item \directlua{myutil.itemprint("4bytes", 2)}

				\uncover<2->{header signature {\small{}(ESC ``Lua'')}}
			\item \directlua{myutil.itemprint("1byte")}

				\uncover<3->{Lua Version

				{\small{}(例では52、つまりLua 5.2)}}
			\item \directlua{myutil.itemprint("1byte")}

				\uncover<4->{Format Version
			
				{\small{}(0 = official(default))}}
			\item \directlua{myutil.itemprint("1byte")}

				\uncover<5->{Endianness flag

				{\small{}(1 = little endian(default), 0 = big endian)}}
			\item \directlua{myutil.itemprint("1byte")}

				\uncover<6->{size of \lstinline{int} {\small{}(C lang)}}
			\item \directlua{myutil.itemprint("1byte")}

				\uncover<7->{size of \lstinline{size_t} {\small{}(C lang)}}
			\item \directlua{myutil.itemprint("1byte")}

				\uncover<8->{size of instruction}
			\item \directlua{myutil.itemprint("1byte")}

				\uncover<9->{size of \lstinline{lua_Number}\footnote[frame]{32bit integer}{\small{}(C lang)}}
			\item \uncover<10->{\textcolor{red}{1byte}

					Integral flag

					{\small{}(0 = floating point(default), 1 = integral number type)}}
		\end{itemize}
	\end{multicols}
\end{frame}
\begin{frame}{\insertsubsubsectionhead}
	\textcolor{gray}{1B 4C 75 61 52 00 01 04 08 04 08 00} \textcolor{red}{19 93 0D 0A 1A 0A}
	\begin{itemize}
		\item 6bytes

		\lstinline{LUAC_DATA}, ``data to catch conversion errors''(ソースのコメントより)

		\begin{itemize}
			\item 19 93

				Lua1.0のリリース年
			\item 0D 0A

				DOSの改行(CR LF)、DOS$\rightarrow$UNIXでの改行のデータ変換の検知
			\item 1A

				SUB
			\item 0A

				UNIXの改行(LF)、UNIX$\rightarrow$DOSでの改行のデータ変換の検知
		\end{itemize}
\end{itemize}
\end{frame}
\begin{frame}{\insertsubsubsectionhead}
	ちなみに

	Lua5.2

	\begin{center}
		1B 4C 75 61 52 00 01 04 08 04 08 00 19 93 0D 0A 1A 0A

		$\Downarrow$
	\end{center}

	Lua5.3

	\begin{center}
		\ltjruby{\textcolor{red}{1B 4C 75 61}}{header signature} \ltjruby{\textcolor{blue}{53}}{version}
		\ltjruby{\textcolor{red}{00}}{format version} \ltjruby{\textcolor{blue}{19 93 0D 0A 1A 0A}}{LUAC\_DATA}
		\ltjruby{\textcolor{red}{04 08 04 08 08}}{objects sizes} \ltjruby{\textcolor{blue}{78 56 00 00 00 00 00 00}}{LUAC\_INT\footnote[frame]{Endiannessの検査、0x5678をdumpしているのでこの場合リトルエンディアン}}
		\ \ltjruby{\textcolor{red}{00 00 00 00 00 00 00 28 77 40}}{LUAC\_NUM\footnote[frame]{IEEE754 float formatの検査}}
	\pause

	\alert{\Huge{}長い}
	\end{center}
\end{frame}
\subsubsection{function block}
\begin{frame}[fragile]{\insertsubsubsectionhead}
	02 00 00 00 02 00 00 00 00 00 02 01 00$\cdots\cdots$

	function block
	\scriptsize
	\begin{multicols}{2}
		\begin{itemize}
			\item \lstinline{int}

				line defined
			\item \lstinline{int}

				last line defined
			\item 1byte

				number of parameters
			\item 1byte

				\lstinline{is_vararg}
			\item 1byte

				number of registers used by the function
			\item List

				\textcolor<2->{red}{number/list of instructions}
			\item List

				\textcolor<2->{red}{number/list of constants}
			\item List

				\textcolor<2->{red}{number/list of upvalues}
			\item List

				debug info
		\end{itemize}
	\end{multicols}
\end{frame}

\section{Profile-Guilded Optimization}
\begin{frame}[t]
  \frametitlesec
  \pause

  \textbf{\scalefont{1.3}\textcolor{red}{PGO}} is \textit{a optimization method}:

  \pause
  \begin{itemize}[<+->]
    \item Uses \textcolor{red}{\it profiling information} of \textcolor{red}{\bf the program execution},
    \item Enables \textcolor{blue}{\it more aggressive optimizations},\\

      such as \textcolor{blue}{\scalefont{1.1}\textbf{inlining}} and \textcolor{blue}{\scalefont{1.1}\textbf{devirtualization}}.
  \end{itemize}
\end{frame}

\section{Function Inlining}
\begin{frame}[t,fragile,containsverbatim]
	\frametitlesec

	\textcolor{highlight}{\textbf{Function inlining}}, or just \textit{inlining}, is\\ the process of \textcolor{subhighlight}{replacing a func call with its body}.

	\pause
	\centering
	\tikzstyle{every picture}+=[remember picture]
	\begin{minipage}[t]{.47\textwidth}
		\begin{onlyenv}<+->
			\begin{minted}[fontsize=\footnotesize]{go}
        func ?\tikz\coordinate(f);?f (x int) int {
          return x * x
        }

        func main() {
          fmt.printf("%d", ?\tikz\coordinate(call);?f(3))
        }
      \end{minted}
		\end{onlyenv}
	\end{minipage}
	\begin{minipage}[t]{.49\textwidth}%
		\vskip-.1\zh
		\begin{uncoverenv}<.(2)->
			\begin{minted}[fontsize=\footnotesize]{go}
        func f(x int) int {
          return x * x
        }

        func main() {
          fmt.printf("%d", ?\tikz\coordinate(inl);\colorbox{orange}{\textcolor{white}{3 * 3}}?)
        }
      \end{minted}
		\end{uncoverenv}
	\end{minipage}

	\begin{tikzpicture}[overlay, every node/.style={inner sep=0pt, outer sep=0pt}]
		\node<+-> [text width=.37\textwidth,text height=2.8\zh,fill=orange,opacity=0.4,anchor=north west] (fentire) at ([xshift=-2.2\zw,yshift=.7\zh] f) {};
		\draw<.-> [boldarrow, draw=blue, bend left=15,opacity=0.5] ([yshift=.5\zh]call) to node[above right, text opacity=1]{\footnotesize \textcolor{highlight}{see definition}} ([yshift=0pt]fentire);

		\draw<+->[boldarrow, draw=orange, bend left=10, opacity=0.6] ([xshift=.2\zw]fentire.east) to node[above right, text opacity=1]{\footnotesize \textcolor{highlight}{inline!}} ([yshift=.5\zh]inl);
	\end{tikzpicture}

	\onslide<+->
	\footnotesize
	\emoji{warning} Note: actually, performed on the IR or more lower level
\end{frame}

\begin{frame}[t]
	\frametitlesec

	Inlining has \textcolor<.-4>{subhighlight}{several benefits}:

	\onslide<+->
	\begin{itemize}[<+->]
		\item[\emoji{thumbs-up}] \textbf<.>{\only<.>{\color{subhighlight}} Reduces \textit{function call overhead}}:

		      \only<.(1)->{\footnotesize}
		      No stack frame setup, no return address, no arguments copying

		\item[\emoji{thumbs-up}] \textbf<.>{\only<.>{\color{subhighlight}}Improves \textit{cache locality}}:

		      \only<.(1)->{\footnotesize}
		      The inlined code is closer to the caller

		\item[\emoji{thumbs-up}] \textbf<.>{\only<.>{\color{subhighlight}}Enables \textit{further optimizations}}:

		      \only<.(1)->{\footnotesize}
		      E.g., \textit<.>{constant propagation}, \textit<.>{dead code elimination}
	\end{itemize}

	\onslide<+->
	\vskip-1.4\zh
	\begin{center}
		\begin{minipage}{.35\textwidth}
			\huge\bf\textcolor{subhighlight}{\scalefont{1.4} BUT$\cdots$}
		\end{minipage}
		\begin{minipage}{.6\textwidth}
			\Large
			\onslide<+->
			There are

			\textcolor{highlight}{\textbf{several conditions}}

			to be applied.
		\end{minipage}
	\end{center}
\end{frame}

\subsection{Conditions for Inlining}
\begin{frame}[t,fragile]
	\frametitlesubs

	\textcolor{highlight}{\textbf{Several conditions}} to be applied:

	\onslide<+->
	\begin{itemize}[<+->]
		% \item[\emoji{check-mark-button}] Specific tags not set,

		% {\small \texttt{//go:noinline}, \texttt{//go:systemstack}, etc.}

		\item[\emoji{check-mark-button}] {\only<.-.(2)>{\color{subhighlight}} \textbf<.-.(2)>{Non-leaf} function:}

		      \onslide<+->
		      \only<.(2)->{\footnotesize}
		      The func \textcolor<.-.(1)>{highlight}{shouldn't} call other funcs.

		      \begin{onlyenv}<.-.(1)>
			      \tikzstyle{every picture}+=[remember picture]
			      \centering
			      \begin{minipage}[t]{.5\textwidth}
				      \begin{minted}[fontsize=\footnotesize]{go}
func f (x int) int {
  return x * x
}

func g (x int) int {
  return f(x)?\tikz\coordinate[anchor=center](g);? + 1
}
              \end{minted}
			      \end{minipage}

			      \onslide<+>
			      \tikz[overlay]{
				      \node[fill=orange,anchor=center,inner sep=2pt, outer sep=2pt] at ([xshift=-.9\zw,yshift=.23\zh]g) {\footnotesize\textcolor{white}{\texttt{f(x)}}};
				      \node[mycallout, callout absolute pointer={($(g)-(.7,.1)$)}, yshift=-2.4\zh,xshift=-5\zw] (c1) at (g) {\textcolor{highlight}{calls \texttt{f}}};
				      \node[draw, xshift=7\zw] (c2) at (c1.east) {\textcolor{highlight}{\textbf{non-leaf function}}};
				      \draw[boldarrow, draw=orange] ([xshift=2pt]c1.east) to ([xshift=-2pt]c2.west);
			      }
		      \end{onlyenv}

		\item[\emoji{check-mark-button}] {\only<.>{\color{subhighlight}}\textbf<.>{Small function}, \textit<.>{"Budget"} $\leq 80$:}

		      \only<+->{
			      \setbeamertemplate{itemize/enumerate subbody begin}{\scriptsize}
			      \footnotesize
		      }

		      Constructs are \textcolor<.(-1)>{subhighlight}{rated} by their cost:\footnote{\url{https://github.com/golang/go/blob/go1.21.0/src/cmd/compile/internal/inline/inl.go}}

		      \begin{itemize}[<1->]
			      \item \textcolor<.(-1)>{subhighlight}{57} for non-leaf func call
			      \item \textcolor<.(-1)>{subhighlight}{1} for \mintinline{go}{panic}
			      \item etc.
		      \end{itemize}

		      The \textit<.(-1)>{budget} is the total cost of func body.

		\item<.-> Other conds$\cdots$
		      \begin{itemize}[<1->]
			      \item Not external function (e.g., C functions),
			      \item No specific tags set,
			      \item Not a complex body, including \mintinline{go}{defer}, \mintinline{go}{recover}, etc.
		      \end{itemize}
	\end{itemize}
\end{frame}

\subsection{Compiler Flags for Inlining}
\begin{frame}[t,fragile]
	\frametitlesubs

	Use \mintinline{sh}{gcflags} to \texttt{go build} to control compiler flags.



	\begin{table}
		\centering
		\begin{tabular}{l|l}
			\hline
			{\semibf Flag} & {\semibf Description} \\
			\hline
			\texttt{-m}    & Verbose optimization  \\
			\hline
			\texttt{-l}    & Control inlining      \\
			\hline
		\end{tabular}
	\end{table}



	% \begin{minted}{go}
	% type Point struct{ x, y int }

	% func (p Point) String() string {
	% return fmt.Sprintf("(%d, %d)", p.x, p.y)
	% }

	% func main() {
	% p := Point{1, 2}
	% fmt.Println(p)
	% }
	% \end{minted}
\end{frame}

\section{Devirtualizing}

\begin{frame}[t]
	\frametitlesec
	\pause
	\textbf{\scalefont{1.3}\textcolor{highlight}{Devirtualizing}} is an optimization method that replaces interface method calls with concrete method calls.
\end{frame}

\section{Profile-Guided Optimization (again)}
\begin{frame}[t]
	\frametitlesec

	\textbf{\scalefont{1.3}\textcolor{highlight}{PGO}} is \textit{an optimization method that}:

	\pause
	\begin{itemize}
		\item Uses \textcolor{highlight}{\it profiling information} from \textcolor{highlight}{\bf program execution},
		\item Enables \textcolor{subhighlight}{\it more aggressive optimizations},\\

		      such as \textcolor{subhighlight}{\scalefont{1.1}\textbf{\tikz\coordinate[remember picture] (inlining);\null inlining}} and \textcolor{subhighlight}{\scalefont{1.1}\textbf{\tikz\coordinate[remember picture] (devirt);\null devirtualization}}.
	\end{itemize}
\end{frame}

\section{宣伝}
\begin{frame}
	\Huge
	\centering

	\scalefont{1.5}
	ここで宣伝です
\end{frame}

\input{src/evaluation.tex}
\input{src/misc.tex}
\section{まとめ}
\begin{frame}
    \frametitlesec

    \begin{itemize}
        \item[\coloremoji{🌋}] Algebraic Effectsが楽しい

            ICFP 2018やML Workshop2018にも\\
            AE関連のトピック%
            \footnote{\url{https://icfp18.sigplan.org/event/mlfamilyworkshop-2018-papers-programming-with-abstract-algebraic-effects}}%
            \footnote{\url{https://icfp18.sigplan.org/event/icfp-2018-papers-versatile-event-correlation-with-algebraic-effects}}

        \item[\coloremoji{🉐}] Algebraic Effects使おう

            \begin{itemize}
                \item[\coloremoji{🛠}] インプリいろいろ
                \item[\coloremoji{💪}] なければ自作も可
            \end{itemize}

        \item[\coloremoji{👨‍💻}] 研究やってます
    \end{itemize}
\end{frame}


\end{document}
