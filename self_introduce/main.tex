\documentclass[aspectratio=169,unicode]{beamer}
\usepackage[ipa,match]{luatexja-preset}
\usepackage{url}
\usepackage{here}
\usepackage{tikz}
\usepackage{luatexja-ruby}
\input{../template/beamertemp}
\input{../template/mylisting}
\setmainjfont{SourceHanSansJP}
\setmainfont{SourceHanSansJP}
\setromanfont{TeXGyreTermes}
\setmonofont{Meslo LG L}
\renewcommand{\familydefault}{\sfdefault}
\renewcommand{\kanjifamilydefault}{\gtdefault}
\defaultfontfeatures{Ligatures=TeX}
\defaultjfontfeatures{BoldFont=SourceHanSansJPBold,Ligatures=TeX}
\jfont{\IPAG}{IPAPGothic:jfm=ujisv} at 1.1\zw
\newcommand{\npause}[1]{\pause\setcounter{lstlisting}{#1}}
% \directlua{myutil = require'codes/texutils/util'}
% \newlength{\bitlen}
% \addtolength{\bitlen}{0.025\textwidth}
% \notoc
\title{自己紹介}
% \subtitle{自己紹介}
\author{河原 悟}
% \subtitle{Tsukuba.pm \#{}3}
% \author{びしょ〜じょ}
\date{May 14, 2016}
\begin{document}
\maketitle
\skipnexttoc
\section{自己紹介}
\skipnexttoc
\subsection{所属}
\skipnexttoc
\begin{frame}
	\frametitlesubs
	\begin{itemize}
		\item 所属: 筑波大学
			\begin{itemize}
				\item 情報学群 情報科学類 4年次3年生
				\item プログラム論理研究室\footnote{\url{http://logic.cs.tsukuba.ac.jp/}} 配属予定
				\item 情報学類誌 WORD 編集部\footnote{\url{http://www.word-ac.net}}

					 \LaTeX、コーラ飲み比べ、楽器制作、漫画紹介など
		\end{itemize}
	\end{itemize}
\end{frame}
\subsection{プログラミング}
\begin{frame}
	\frametitlesubs
	\begin{itemize}
		\item プログラミング歴: 3年

			大学に入ってから
			\begin{itemize}
				\item GitHub ID: Nymphium
				\item \structure{Lua}

					Lua製OSSにいくつかPRを投げる、MLに目を通す
				\item \alert{MoonScript}

					日本人唯一のコントリビューター
			\end{itemize}
	\end{itemize}
\end{frame}
\section{活動}
\subsection{アルバイト}
\begin{frame}
	\frametitlesubs

	\begin{itemize}
		\item 産業総合研究所
			\begin{itemize}
				\item 2014/9 〜 2015/2
				\item Java, Shell Script
		\end{itemize}
		\item アジルポイント
			\begin{itemize}
				\item 2015/05 〜
				\item JavaScript(jQuery), 英日翻訳
			\end{itemize}
	\end{itemize}
\end{frame}
\subsection{contribute for MoonScript}
\begin{frame}
	\frametitlesubs
	\begin{itemize}
		\item MoonScript? \footnote{\url{https://github.com/leafo/moonscript}}
			\begin{itemize}
				\item Luaにトランスパイル可能なスクリプト言語

					\begin{itemize}
						\item CoffeeScriptライクな構文
						\item トランスパイラ/インタプリタはMoonScriptで記述されている
					\end{itemize}
			\end{itemize}\pause

		\item ビルドシステム周り、構文のバグフィックスなど
		\item {\footnotesize{}\lstinline{curl https://github.com/leafo/moonscript/blob/master/CHANGELOG.md | grep nymphium}}
		\item \alert{日本人唯一のコントリビューター}
		\item 3番めくらいにコード変更量が多い!!
\end{itemize}\end{frame}
\subsection{他}
\begin{frame}
	\frametitlesubs
	\begin{itemize}
		\item MoonScriptによるLua処理系の実装\footnote{\url{http://nymphium.github.io/pdf/information_special_seminar.html}}
			\begin{itemize}
				\item CPSによる例外処理機構を実装
			\end{itemize}
		\item Optimizer for Lua VM (鋭意製作中)
			\begin{itemize}
				\item Lua VMの実行形式を読み取り、命令列を高速になるように最適化していく
			\end{itemize}
		\item MoonScript開発用ツール
			\begin{itemize}
				\item REPL\footnote{\url{https://github.com/nymphium/moor}}
				\item Syntastic用のLint\footnote{\url{https://github.com/nymphium/syntastic-moonscript}}
			\end{itemize}
	\end{itemize}
\end{frame}
\end{document}
