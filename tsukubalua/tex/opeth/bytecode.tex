\subsection{Bytecode}
\begin{frame}
	\frametitlesubs
	\begin{figure}
		\centering
		\begin{tikzpicture}
			\node[text width=13\zw] (sousa) at (0, 0) {Lua VM 5.3のバイトコード\\を操作したい};
			\node[right = .5 of sousa] (arrow) {$\Rightarrow$};
			\node[right = .5 of arrow, text width=13\zw] {バイトコードのdocumentは\\\alert<2->{ない}};
		\end{tikzpicture}
	\end{figure}

	\pause
	\vspace{3\zw}
	\begin{center}
		$\Rightarrow$ \alert{自分で読み解くしかない}
	\end{center}
\end{frame}
\begin{frame}
	\frametitlesubs

	有志の非公式ドキュメント
	\begin{itemize}
		\item Lua VM 5.3 instructions (bytecodeではない)\footnote{\url{https://github.com/dibyendumajumdar/ravi/blob/master/readthedocs/lua_bytecode_reference.rst}}
		\item Lua VM 5.1 reference\footnote{\url{http://luaforge.net/docman/83/98/ANoFrillsIntroToLua51VMInstructions.pdf}}
	\end{itemize}
	\pause

	Lua VM bytecodeを読むためのツール
	\begin{itemize}
		\item \lstinline{luac -l -l luac.out}
		\item \lstinline{xxd -g 1 luac.out | nvim - -R}\pause
		\item ソースコード\footnote{\url{https://www.lua.org/source}}
	\end{itemize}

	\pause
	\begin{center}
		簡単に言うと\alert{\Huge{}気合}
	\end{center}
\end{frame}
\begin{frame}[fragile]
	\frametitlesubs
	\hspace{-.5\zw}
	\begin{minipage}[t]{.34\textwidth}
		\bgroup\footnotesize
		\begin{lstlisting}[numbers=none,language={[5.3]lua}]
print("hello, world!")
		\end{lstlisting}
		\tiny
		\begin{lstlisting}[numbers=none]
 $ luac -l -l luac.out

main <hello.lua:0,0> (4 instructions at 0x16e79e0)
0+ params, 2 slots, 1 upvalue, 0 locals, 2 constants, 0 functions
  1  [1]  GETTABUP     0 0 -1  ; _ENV "print"
  2  [1]  LOADK        1 -2    ; "hello, world!"
  3  [1]  CALL         0 2 1
  4  [1]  RETURN       0 1
constants (2) for 0x16e79e0:
  1  "print"
  2  "hello, world!"
locals (0) for 0x16e79e0:
upvalues (1) for 0x16e79e0:
  0  _ENV    1       0
		\end{lstlisting}
		\egroup
	\end{minipage}
	\begin{minipage}{.01\textwidth}
		\ 
	\end{minipage}
	\begin{minipage}[t]{.64\textwidth}
		\pause
		\bgroup\tiny
		\begin{lstlisting}[numbers=none]
$ xxd -g 1 luac.out
00000000: 1b 4c 75 61 53 00 19 93 0d 0a 1a 0a 04 08 04 08  .LuaS...........
00000010: 08 78 56 00 00 00 00 00 00 00 00 00 00 00 28 77  .xV...........(w
00000020: 40 01 0b 40 68 65 6c 6c 6f 2e 6c 75 61 00 00 00  @..@hello.lua...
00000030: 00 00 00 00 00 00 02 02 04 00 00 00 06 00 40 00  ..............@.
00000040: 41 40 00 00 24 40 00 01 26 00 80 00 02 00 00 00  A@..$@..&.......
00000050: 04 06 70 72 69 6e 74 04 0e 68 65 6c 6c 6f 2c 20  ..print..hello,
00000060: 77 6f 72 6c 64 21 01 00 00 00 01 00 00 00 00 00  world!..........
00000070: 04 00 00 00 01 00 00 00 01 00 00 00 01 00 00 00  ................
00000080: 01 00 00 00 00 00 00 00 01 00 00 00 05 5f 45 4e  ............._EN
00000090: 56                                                                  V
		\end{lstlisting}
		\egroup\normalfont
		\pause
		\begin{center}
			\vspace{1\zw}
			\LARGE
			\alert{???}
		\end{center}
	\end{minipage}
\end{frame}
\begin{frame}[fragile]
	\frametitlesubs
	\begin{center}
		\begin{minipage}[t]{.4\textwidth}
			\bgroup\tiny
			\begin{lstlisting}[numbers=none]
1b 4c 75 61 53 00 19 93 0d 0a 1a 0a 04 08 04 08
08 78 56 00 00 00 00 00 00 00 00 00 00 00 28 77
40 01 0b 40 68 65 6c 6c 6f 2e 6c 75 61 00 00 00
00 00 00 00 00 00 02 02 04 00 00 00 06 00 40 00
41 40 00 00 24 40 00 01 26 00 80 00 02 00 00 00
04 06 70 72 69 6e 74 04 0e 68 65 6c 6c 6f 2c 20
77 6f 72 6c 64 21 01 00 00 00 01 00 00 00 00 00
04 00 00 00 01 00 00 00 01 00 00 00 01 00 00 00
01 00 00 00 00 00 00 00 01 00 00 00 05 5f 45 4e
56
			\end{lstlisting}
			\egroup
		\end{minipage}
	\end{center}
	\pause
	\begin{figure}[h]
		\def\position{(0.1, -1.76)}
		\tikzset{just here/.style = {above right,inner sep=0mm}}
		\centering
		\begin{tikzpicture}[remember picture]
			\useasboundingbox (0.1,-1.76);
			\coordinate (top) at (-3, 2.3);
			\node[above right = .2 of top] (label) {\textcolor{blue}{header block}};
			\coordinate[right = 6.2 of top] (p1);
			\coordinate[below = 0.55 of p1] (p2);
			\coordinate[left = 5.4 of p2] (p3);
			\coordinate[below = 0.28 of p3] (p4);
			\coordinate[left = 0.8 of p4] (p5);
			\draw[blue, dashed, very thick] (top) -- (p1) -- (p2) -- (p3) -- (p4) -- (p5) -- (top);
		\end{tikzpicture}
	\end{figure}
	\pause
	\begin{figure}[h]
		\tikzset{just here/.style = {above right,inner sep=0mm}}
		\centering
		\begin{tikzpicture}[remember picture]
			\useasboundingbox (0.1,-1.76);
			\coordinate (bottom) at (-3, 1.33);
			\node[below right = .2 of bottom] (label) {\textcolor{red}{function block}};
			\coordinate[right = 6.2 of bottom] (p1);
			\coordinate[above = 2.14 of p1] (p2);
			\coordinate[left = 5.4 of p2] (p3);
			\coordinate[below = .28 of p3] (p4);
			\coordinate[left = .8 of p4] (p5);
			\draw[red, dashed, very thick] (bottom) -- (p1) -- (p2) -- (p3) -- (p4) -- (p5) -- (bottom);
		\end{tikzpicture}
	\end{figure}
\end{frame}

