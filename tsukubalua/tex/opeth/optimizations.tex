\subsection{Optimizations}
\begin{frame}
	\frametitlesubs
	\begin{itemize}
		\item Constant Folding
		\item Constant Propagation
		\item Dead-Code Elimination
		\item Function Inlining
		\item Unreachable Block Removal
		\item Unused Resource Removal
	\end{itemize}
\end{frame}
\subsubsection{Constant Folding}
\begin{frame}[fragile]
	\frametitle{Constant Folding}

	\begin{enumerate}
		\item 演算命令のオペランドの型を調べて
		\item \lstinline{table}、\lstinline{userdata}\alert<2->{以外なら} \uncover<3->{$\Leftarrow$ \structure{メタメソッド}を考慮}
		\item 値を取ってきて
		\item 演算をおこない
		\item 即値命令にswap
	\end{enumerate}
\end{frame}
\subsubsection{Constant Propagation}
\begin{frame}[fragile]
	\frametitle{Constant Propagation}

	\begin{enumerate}
		\item \lstinline{MOVE}命令が参照してるregisterの定義位置を見て
		\item \lstinline{LOADK}なら\lstinline{MOVE}を\lstinline{LOADK}にする
	\end{enumerate}

	\pause
	\begin{itemize}
		\item 単体では速度改善なさそう
		\item \lstinline{LOADK}への依存が減るので、他の最適化を有利に進められる
		\item[] <3-> \textcolor{gray}{(今回の実装では)いまいちぱっとしない}
	\end{itemize}
\end{frame}
\subsubsection{Dead-Code Elimination}
\begin{frame}[fragile]
	\frametitle{Dead-Code Elimination}
	\begin{enumerate}
		\item \lstinline{LOADK}、\lstinline{MOVE}、\lstinline{CLOSURE}、\lstinline{LOADNIL}が生成するregistrの使用を調べ
		\item 0個の場合命令を消す
	\end{enumerate}

	\pause
	\begin{itemize}
		\item DU/UD Chainのわかりやすい使用例
	\end{itemize}
\end{frame}
\subsubsection{Function Inlining}
\begin{frame}[fragile]
	\frametitle{Function Inlining}
	\begin{enumerate}
		\item \lstinline{CALL}命令が引っ張ってくるclosureを見て
		\item 再帰関数でなければ展開
	\end{enumerate}
	\pause
	\begin{itemize}
		\item register windowの使用を抑えられる
		\item<3-> \alert{実は頼みの綱}
		\item<4-> バグがヤバい

			ア
	\end{itemize}
\end{frame}
\subsubsection{Unreachable Block Removal}
\begin{frame}
	\frametitle{Unreachable Block Removal}
	\begin{enumerate}
		\item 後続ブロックを持たない基本ブロックを丸々削除
		\item だけ
	\end{enumerate}
	\begin{itemize}
		\item 速くはならないがバイトコードのサイズ縮小に貢献
	\end{itemize}
\end{frame}
\subsubsection{Unused Resource Removal}
\begin{frame}
	\frametitle{Unused Resource Removal}
	\begin{enumerate}
		\item constant list、prototype listから不要なものを削除
		\item だけ
	\end{enumerate}
	\begin{itemize}
		\item 速くはならないがバイトコードのサイズ縮小に貢献
	\end{itemize}
\end{frame}

