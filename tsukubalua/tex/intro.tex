\section{Intro}
\begin{frame}
	\frametitlesec
	こんにちは、びしょ〜じょです。
	\begin{itemize}
		\item ここの大学に4年滞在中の3年生
		\item {\alert{Lua/MoonScript}をよく書く}
		\item ライトなメタラー
	\end{itemize}
\end{frame}
\begin{frame}
	\frametitle{流れ}
	\begin{enumerate}
		\item tsukuba.pmというイベントでLuaのバイトコード解析\footnote{\url{http://nymphium.github.io/pdf/tsukubapm3-luavm.html}}
		\item あまり最適化されてないことが判明
		\item \alert{optimizer作るか}
	\end{enumerate}
\end{frame}
\subsection{Lua is $\dots\dots$}
\begin{frame}
	\frametitlesubs
	\begin{itemize}
		\item 弱い動的型付けなスクリプト言語
		\item 文法が簡単、予約語も22個と少ない
		\pause
		\item 関数がファーストクラス
			\begin{itemize}
				\item ナウい関数型プログラミングも可能
			\end{itemize}
		\pause
		\item \alert{唯一}のデータ構造 table
			\begin{itemize}
				\item 簡単に言うと\alert{連想配列}
				\item オブジェクトは全部キーにも要素にも
				\item \structure{メタテーブル}で色々拡張
			\end{itemize}
	\end{itemize}
\end{frame}
\subsection{Implementations}
\begin{frame}
	\frametitlesubs
	\begin{itemize}
		\item \alert<2->{(PUC-Lua)}
			\begin{itemize}
				\item リオデジャネイロ・カトリカ大学開発の、いわゆる本家
				\item 軽量、組み込みで広く活躍
			\end{itemize}
		\item \alert<2->{LuaJIT}
			\begin{itemize}	
				\item だいぶ速い。FFIモジュールなども提供
		\end{itemize}
		\item LuaJ、Rembulan
			\begin{itemize}
				\item JVM実装。
		\end{itemize}
		\item その他色々
	\end{itemize}
\end{frame}
\subsubsection{Other}
\begin{frame}
	\frametitlesubs
	その他
	\begin{itemize}
		\item llix
			\begin{itemize}
				\item 拙作。例外処理構文を追加
			\end{itemize}
		\item TypedLua
			\begin{itemize}
				\item 型アノテーション、型定義ファイルなど。トランスパイラ
				\item GSoCで募集してたり\footnote{\url{https://summerofcode.withgoogle.com/archive/2016/organizations/4733835644239872/}}
			\end{itemize}
		\item Ravi
			\begin{itemize}
				\item LLVM+Luaの文法+$\alpha$。別言語
			\end{itemize}
		\item Terra
			\begin{itemize}
				\item multi-stage programming
			\end{itemize}
		\item MoonScript
			\begin{itemize}
				\item altLua的なモノ。\pause{}ちょっとコントリビュート
			\end{itemize}
	\end{itemize}
\end{frame}
\subsection{About Lua VM}
\begin{frame}
\frametitlesubs
\begin{itemize}
	\item PUC-Lua
	\item \alert{レジスターベース} (Lua 5.0〜)
	\item \structure{関数呼び出しはレジスターウィンドウ}
	\item 47個の命令 (Lua 5.3)
\end{itemize}
\end{frame}
\begin{frame}[fragile]
\frametitlesec
\alert{Lua VM}, register-based Virtual Machine
\begin{figure}[H]
	\bgroup
	\footnotesize\tt
	\tikzset{
		LL/.style={
			draw=black,decorate,
			decoration={snake, segment length=3mm,post}
		},
		every node/.style = {draw, align=center}
	}

	\begin{tikzpicture}
		\node[ellipse,fill=\nodecolor] (source) {Lua source};
		\node[right = 2 of source, ellipse,fill=\nodecolor] (bytecode) {\textcolor<2>{red}{bytecode}};
		\node[draw=none, above] at ($(source)!0.5!(bytecode)$) {\textrm{compile}};
		\node[draw=none,right = of bytecode] (run) {(run on the VM)};
		\draw[->] (source) -- (bytecode);
		\draw[->] (bytecode) -- (run);
	\end{tikzpicture}
	\egroup
\end{figure}
\pause
\pause

積極的に最適化が\structure{行われない}

\bgroup
\small
\begin{minipage}{.1\textwidth}
\ 
\end{minipage}
\begin{minipage}{.3\textwidth}
\begin{lstlisting}[language={[5.3]lua}]
local x = 3
local y = 5
print(x + y)\end{lstlisting}
\end{minipage}
\begin{minipage}{.2\textwidth}
\begin{center}
\structure{$\Rightarrow$}

compile
\end{center}
\end{minipage}
\begin{minipage}{.35\textwidth}
	\begin{lstlisting}
LOADK       0   0
LOADK       1   1
GETTABUP   2    0   -3
ADD         3   0    1
CALL        2   2    1
RETURN      0   1\end{lstlisting}
\end{minipage}
\begin{minipage}{.1\textwidth}
\ 
\end{minipage}
\egroup

\only<4->{%
\vskip-3.3\zw%
\hspace{10\zw}%
\fcolorbox{black}{white}{\parbox{\dimexpr10\zw\fboxsep-2\fboxrule\relax}{\alert{コンパイル時に値が\ \ $\uparrow$ 分かる(定数化可能)}}}%
\vskip.4\zw%
}

\only<5->{%
\vskip-8.5\zw%
\hskip7.7\zw%
\fcolorbox{black}{white}{\parbox{\dimexpr11\zw\fboxsep-2\fboxrule\relax}{\alert{足し算の結果が分かればこの定数はいらない\ \ $\rightarrow$}}}%
\vskip5.7\zw%
}
\end{frame}

