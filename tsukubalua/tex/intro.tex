\section{intro}
\begin{frame}
	こんにちは、びしょ〜じょです。
	\begin{itemize}
		\item ここの大学に4年滞在中の3年生
		\item {\alert{Lua/MoonScript}をよく書く}
		\item ライトなメタラー
	\end{itemize}
	\begin{figure}[H]
		\centering
		\includegraphics[height=.3\textheight]{img/dousei_wedding_women.png}
		\caption{ウェディングドレスを着た同性愛者の女性カップルが結婚式を挙げているイラストです。}
	\end{figure}
\end{frame}
\begin{frame}
	\frametitle{流れ}
	\begin{enumerate}
		\item tsukuba.pmというイベントでLuaのバイトコード解析\footnote{\url{http://nymphium.github.io/pdf/tsukubapm3-luavm.html}}
		\item あまり最適化されてないことが判明
		\item \alert{optimizer作るか}
	\end{enumerate}
\end{frame}
\begin{frame}[fragile]
\frametitlesec
\alert{Lua VM}, register-based Virtual Machine
\begin{figure}[H]
	\bgroup
	\footnotesize\tt
	\tikzset{
		LL/.style={
			draw=black,decorate,
			decoration={snake, segment length=3mm,post}
		},
		every node/.style = {draw, align=center}
	}

	\begin{tikzpicture}
		\node[ellipse,fill=\nodecolor] (source) {Lua source};
		\node[right = 2 of source, ellipse,fill=\nodecolor] (bytecode) {\textcolor<2>{red}{bytecode}};
		\node[draw=none, above] at ($(source)!0.5!(bytecode)$) {\textrm{compile}};
		\node[draw=none,right = of bytecode] (run) {(run on the VM)};
		\draw[->] (source) -- (bytecode);
		\draw[->] (bytecode) -- (run);
	\end{tikzpicture}
	\egroup
\end{figure}
\pause
\pause

積極的に最適化が\structure{行われない}

\bgroup
\small
\begin{minipage}{.1\textwidth}
\ 
\end{minipage}
\begin{minipage}{.3\textwidth}
\begin{lstlisting}[language={[5.3]lua}]
local x = 3
local y = 5
print(x + y)\end{lstlisting}
\end{minipage}
\begin{minipage}{.2\textwidth}
\begin{center}
\structure{$\Rightarrow$}

compile
\end{center}
\end{minipage}
\begin{minipage}{.35\textwidth}
	\begin{lstlisting}
LOADK       0   0
LOADK       1   1
GETTABUP   2    0   -3
ADD         3   0    1
CALL        2   2    1
RETURN      0   1\end{lstlisting}
\end{minipage}
\begin{minipage}{.1\textwidth}
\ 
\end{minipage}
\egroup

\only<4->{%
\vskip-3.3\zw%
\hspace{10\zw}%
\fcolorbox{black}{white}{\parbox{\dimexpr10\zw\fboxsep-2\fboxrule\relax}{\alert{コンパイル時に値が\ \ $\uparrow$ 分かる(定数化可能)}}}%
\vskip.4\zw%
}

\only<5->{%
\vskip-8.5\zw%
\hskip7.7\zw%
\fcolorbox{black}{white}{\parbox{\dimexpr11\zw\fboxsep-2\fboxrule\relax}{\alert{足し算の結果が分かればこの定数はいらない\ \ $\rightarrow$}}}%
\vskip5.7\zw%
}
\end{frame}

