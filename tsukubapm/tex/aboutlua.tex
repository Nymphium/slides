\subsection{Lua/MoonScriptとは}
\begin{frame}
	\frametitlesubs
	\begin{itemize}
		\item \alert{Lua}

			弱い動的型付けなスクリプト言語 \uncover<2->{\textcolor{gray}{$\leftarrow${\Huge{}Perlでは$\cdots\cdots${}?}}}\\
			ブラジル産!! 軽量!! 関数もファーストクラスで扱えるぞ!!

			文法が簡単、予約語も22個と少ない
		\item \structure{MoonScript}

			JSに対するCoffeeScriptみたいなやつ

			\lstinline{end}地獄の解消、リスト内包表記、クラスベースOOPなど文法の強化
	\end{itemize}
\end{frame}
\begin{frame}
	\frametitlesubs
	\textcolor{gray}{{\Huge{}Perlでは$\cdots\cdots${}}}
\end{frame}
\subsection{Luaとは?!}
\begin{frame}
	\frametitlesubs
	\begin{itemize}
		\item \structure{軽量}

			\uncover<2->{アイマス2からNetBSD、Wiresharkや\structure{このスライド}(\textrm{Lua\LaTeX{}})にまで組み込める}

			\uncover<3->{\alert{レジスタベースVM上で動く} (PUC-Lua)}
		\item \structure{関数がファーストクラス}

			\uncover<4->{高階関数を扱うことができ、CPSを用いた例外処理機構のあるLuaインタプリタ\footnote[frame]{\url{https://github.com/nymphium/llix}}も実装できる}
		\item \alert{唯一}のデータ構造 table
			\uncover<5->{
				\begin{itemize}
					\item 簡単に言うと\alert{連想配列}\only<6->{\textcolor{gray}{$\leftarrow${\Huge{}Perlでは$\cdots\cdots${}?}}}\only<1-5>{、数字もキー}
					\item 関数、数値、文字列、table自身も、オブジェクトは全部詰め込める
					\item \structure{メタテーブル}という機能を用いることにより演算子オーバーロードやプロトタイプベースのオブジェクト指向プログラミンができる
				\end{itemize}}
	\end{itemize}
\end{frame}
\begin{frame}
	\frametitlesubs
	\textcolor{gray}{{\Huge{}Perlでは$\cdots\cdots${}\pause{}ない}}
\end{frame}
