\begin{frame}
	\frametitlesec

	かいた

	{\tiny\url{https://gist.github.com/Nymphium/d47385929fd0d23e98207670f9c588c3}}
	\begin{itemize}
		\item Lua 5.2バイトコードをターゲット
		\item header/function blockのconstantsまで解析
		\item MoonScriptで実装

			\textcolor{gray}{\tiny{}バグを一つみつけてしまった$\cdots\cdots$}
		\item 5.1のバイトコードの資料見ながら実装していったら5.2とちょこちょこ変わっててホンマキレタ
		\item 5.2と5.3もちょいちょい違っててキレタ(2)
	\end{itemize}
\end{frame}
\begin{frame}
	\frametitlesec
	\begin{center}
		\Huge{}デモ
	\end{center}
	\tiny\directlua{dofile"codes/texutils/dumpdemo.lua"}

	\begin{figure}[h]
		\tikzset{just here/.style = {above right,inner sep=0mm}}
		\centering
		\begin{tikzpicture}[remember picture]
			\useasboundingbox (0,0);
			\draw[red, dashed, very thick] (-4.8, 1.4) -- (-4.8, 1.17) -- (-7.1, 1.17) -- (-7.1, .84) -- (4.7, .84) -- (4.7, 1.4) -- (-4.8, 1.4);
		\end{tikzpicture}
	\end{figure}\normalsize

	このへん(function block)を読ませる
\end{frame}
\begin{frame}[fragile]
	\frametitlesec

	できそうなこと
	\begin{itemize}
		\item 最適化

			最適化っぽいことはほとんどしない(\lstinline{TAICALL}、numberの即値演算程度)のでいっぱいできそう
		\item 型チェック

			constants poolから逆に命令を見ていけばいけるか
		\item 可視化 (フローグラフなど)
	\end{itemize}
\end{frame}
