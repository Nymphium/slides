\subsection{Lua VMについて}
\begin{frame}
\frametitlesubs
\begin{itemize}
	\item リオデジャネイロ・カトリカ大学の開発したLua処理系(つまり本家, PUC-Lua)のVM

		\begin{center}
			\textcolor<2->{gray}{ソース$\rightarrow${\color<2->{red}{}バイトコード}$\rightarrow$VMで実行}
		\end{center}
\end{itemize}
\vspace{-1\zw}
\begin{center}
	\only<2->{\alert{の調査と解析}}
	% \alert{\only<2,4>{\only<4>{\hspace{6.5\zw}}の調査}\only<3,4>{\only<3>{\hspace{10.2\zw{}の}}{\only<4>{と}デモ}}}
	\vspace{1\zw}

	\only<3>{\alert{Lua5.2を見る (Lua\LaTeX{}がv5.2ベースなので)}}
\end{center}
\end{frame}
\begin{frame}[fragile]
	\frametitlesubs
	1環境(メインチャンク、関数クロージャ)ごとにレジスタ(slots, constants、upvalues、locals、functions)を用意

	\begin{lstlisting}[language={[5.2]lua},numbers=none]
function hello()
	print("hello")
end\end{lstlisting}

\begin{table}[h]
	\centering
	\begin{tabular}{rcl}
		slots: &2個&(\lstinline|print|と\lstinline|"hello"|をセットするレジスタ)\\
		constants:& 2個&(\lstinline|"print"|、\lstinline|"hello"|)\\
		upvalues:& 1個&(\lstinline|_ENV|\footnote[frame]{グローバル変数は\lstinline|_ENV| tableから持ってくるため})\\
		functions:& 0個&\\
		locals:& 0個&
	\end{tabular}
\end{table}
\end{frame}
