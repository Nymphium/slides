\subsection{バイトコード}
\begin{frame}[fragile]
	\frametitlesubs
	\begin{itemize}
		\item \lstinline{luac}コマンドでバイトコードを生成
		\item \lstinline{luac -l bytecode.out}でいい感じに出力
		\item \lstinline{luac -l -l bytecode.out}で情報が増える
	\end{itemize}
\end{frame}
\begin{frame}[fragile]
	\frametitlesubs
ひとまず標準入力から渡してみる
	\small
	\begin{lstlisting}[numbers=none,language=sh]
cat <<LUACODE | luac -l -
print("hello, world")
LUACODE
	\end{lstlisting}
	\pause
	\begin{center}$\Downarrow$\end{center}
	\begin{lstlisting}[numbers=none]
main <stdin:0,0> (4 instructions at 0x2268790)
0+ params, 2 slots, 1 upvalue, 0 locals, 2 constants, 0 functions
        1       [1]     GETTABUP        0 0 -1  ; _ENV "print"
        2       [1]     LOADK           1 -2    ; "hello, world"
        3       [1]     CALL            0 2 1
        4       [1]     RETURN          0 1\end{lstlisting}
\end{frame}
\begin{frame}[fragile]
	\frametitlesubs
	\textrm{Lua\LaTeX}にダイレクトに書いてみる
	\bgroup
	\scriptsize
	\mylisting[language={[5.3]lua}]{codes/dumpdemo.lua}
	\npause{0}
	\tiny
	\directlua{dofile("codes/dumpdemo.lua")}
	\egroup
\end{frame}
\subsection{読み方}
\begin{frame}[fragile]
	\frametitlesubs
	\tiny
	\directlua{dofile("codes/dumpdemo.lua")}
	\normalsize

	まずこれを読む。
\end{frame}
\subsubsection{header block}
\begin{frame}[fragile]{\insertsubsubsectionhead}
	\directlua{dofile("codes/headerprint.lua")}

	header block
	\begin{multicols}{2}
		\begin{itemize}
			\item \directlua{myutil.itemprint("4bytes", 2)}

				\uncover<2->{header signature {\small{}(ESC ``Lua'')}}
			\item \directlua{myutil.itemprint("1byte")}

				\uncover<3->{Lua Version

				{\small{}(例では52、つまりLua 5.2)}}
			\item \directlua{myutil.itemprint("1byte")}

				\uncover<4->{Format Version
			
				{\small{}(0 = official(default))}}
			\item \directlua{myutil.itemprint("1byte")}

				\uncover<5->{Endianness flag

				{\small{}(1 = little endian(default), 0 = big endian)}}
			\item \directlua{myutil.itemprint("1byte")}

				\uncover<6->{size of \lstinline{int} {\small{}(C lang)}}
			\item \directlua{myutil.itemprint("1byte")}

				\uncover<7->{size of \lstinline{size_t} {\small{}(C lang)}}
			\item \directlua{myutil.itemprint("1byte")}

				\uncover<8->{size of instruction}
			\item \directlua{myutil.itemprint("1byte")}

				\uncover<9->{size of \lstinline{lua_Number}\footnote[frame]{32bit integer}{\small{}(C lang)}}
			\item \uncover<10->{\textcolor{red}{1byte}

					Integral flag

					{\small{}(0 = floating point(default), 1 = integral number type)}}
		\end{itemize}
	\end{multicols}
\end{frame}
\begin{frame}{\insertsubsubsectionhead}
	\textcolor{gray}{1B 4C 75 61 52 00 01 04 08 04 08 00} \textcolor{red}{19 93 0D 0A 1A 0A}
	\begin{itemize}
		\item 6bytes

		\lstinline{LUAC_DATA}, ``data to catch conversion errors''(ソースのコメントより)

		\begin{itemize}
			\item 19 93

				Lua1.0のリリース年
			\item 0D 0A

				DOSの改行(CR LF)、DOS$\rightarrow$UNIXでの改行のデータ変換の検知
			\item 1A

				SUB
			\item 0A

				UNIXの改行(LF)、UNIX$\rightarrow$DOSでの改行のデータ変換の検知
		\end{itemize}
\end{itemize}
\end{frame}
\begin{frame}{\insertsubsubsectionhead}
	ちなみに

	Lua5.2

	\begin{center}
		1B 4C 75 61 52 00 01 04 08 04 08 00 19 93 0D 0A 1A 0A

		$\Downarrow$
	\end{center}

	Lua5.3

	\begin{center}
		\ltjruby{\textcolor{red}{1B 4C 75 61}}{header signature} \ltjruby{\textcolor{blue}{53}}{version}
		\ltjruby{\textcolor{red}{00}}{format version} \ltjruby{\textcolor{blue}{19 93 0D 0A 1A 0A}}{LUAC\_DATA}
		\ltjruby{\textcolor{red}{04 08 04 08 08}}{objects sizes} \ltjruby{\textcolor{blue}{78 56 00 00 00 00 00 00}}{LUAC\_INT\footnote[frame]{Endiannessの検査、0x5678をdumpしているのでこの場合リトルエンディアン}}
		\ \ltjruby{\textcolor{red}{00 00 00 00 00 00 00 28 77 40}}{LUAC\_NUM\footnote[frame]{IEEE754 float formatの検査}}
	\pause

	\alert{\Huge{}長い}
	\end{center}
\end{frame}
\subsubsection{function block}
\begin{frame}[fragile]{\insertsubsubsectionhead}
	02 00 00 00 02 00 00 00 00 00 02 01 00$\cdots\cdots$

	function block
	\scriptsize
	\begin{multicols}{2}
		\begin{itemize}
			\item \lstinline{int}

				line defined
			\item \lstinline{int}

				last line defined
			\item 1byte

				number of parameters
			\item 1byte

				\lstinline{is_vararg}
			\item 1byte

				number of registers used by the function
			\item List

				\textcolor<2->{red}{number/list of instructions}
			\item List

				\textcolor<2->{red}{number/list of constants}
			\item List

				\textcolor<2->{red}{number/list of upvalues}
			\item List

				debug info
		\end{itemize}
	\end{multicols}
\end{frame}
