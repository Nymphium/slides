% \documentclass[aspectratio=169,unicode,handout]{beamer}
\documentclass[aspectratio=169,unicode]{beamer}
\usepackage[ipa,match]{luatexja-preset}
\usepackage{url}
\usepackage{here}
\usepackage{tikz}
\usepackage{luatexja-ruby}
\input{../template/beamertemp}
\input{../template/mylisting}
\setmainjfont{SourceHanSansJP}
\setmainfont{SourceHanSansJP}
\setromanfont{TeXGyreTermes}
\setmonofont{Meslo LG L}
\renewcommand{\familydefault}{\sfdefault}
\renewcommand{\kanjifamilydefault}{\gtdefault}
\defaultfontfeatures{Ligatures=TeX}
\defaultjfontfeatures{BoldFont=SourceHanSansJPBold,Ligatures=TeX}
\jfont{\IPAG}{IPAPGothic:jfm=ujisv} at 1.1\zw
\newcommand{\npause}[1]{\pause\setcounter{lstlisting}{#1}}
\directlua{myutil = require'codes/texutils/util'}
\newlength{\bitlen}
\addtolength{\bitlen}{0.025\textwidth}
\title{LuaのVM}
\subtitle{Tsukuba.pm \#{}3}
\author{びしょ〜じょ}
\date{May 14, 2016}
\begin{document}
\maketitle
\skipnexttoc
\section{自己紹介}
\section{Today's Topic}
\begin{frame}
  \frametitlesec

  By the way, do you...

  \large
  \begin{itemize}[<+->]
    \pause
    \item[\emoji{woman-raising-hand}] Use Go Compiler?

    \item[\large \emoji{woman-raising-hand}] {\large Use Optimization?}

    \item[] Then,

      \onslide<+->{
        \huge\vspace*{0.5\zh}\emoji{woman-raising-hand}\vspace*{-2\zh}
        \begin{center}
          \color{red}
          Use

          Profile-Guilded

          Optimization?
        \end{center}
      }
  \end{itemize}
\end{frame}

\begin{frame}
  \frametitlesec
  \Large
  \semibf

  \centering

  Learn about\pause

  \vspace*{15pt}
  {\scalefont{1.8}\color{red}\textbfslant{Profile-Guilded Optimization}}\pause
  \vspace*{6pt}

  and
  \vspace*{10pt}

  around \textcolor{blue}{\scalefont{1.5}its optimizations}

\end{frame}

\section{Luaとは}
\subsection{Lua/MoonScriptとは}
\begin{frame}
	\frametitlesubs
	\begin{itemize}
		\item \alert{Lua}

			弱い動的型付けなスクリプト言語 \uncover<2->{\textcolor{gray}{$\leftarrow${\Huge{}Perlでは$\cdots\cdots${}?}}}\\
			ブラジル産!! 軽量!! 関数もファーストクラスで扱えるぞ!!

			文法が簡単、予約語も22個と少ない
		\item \structure{MoonScript}

			JSに対するCoffeeScriptみたいなやつ

			\lstinline{end}地獄の解消、リスト内包表記、クラスベースOOPなど文法の強化
	\end{itemize}
\end{frame}
\begin{frame}
	\frametitlesubs
	\textcolor{gray}{{\Huge{}Perlでは$\cdots\cdots${}}}
\end{frame}
\subsection{Luaとは?!}
\begin{frame}
	\frametitlesubs
	\begin{itemize}
		\item \structure{軽量}

			\uncover<2->{アイマス2からNetBSD、Wiresharkや\structure{このスライド}(\textrm{Lua\LaTeX{}})にまで組み込める}

			\uncover<3->{\alert{レジスタベースVM上で動く} (PUC-Lua)}
		\item \structure{関数がファーストクラス}

			\uncover<4->{高階関数を扱うことができ、CPSを用いた例外処理機構のあるLuaインタプリタ\footnote[frame]{\url{https://github.com/nymphium/llix}}も実装できる}
		\item \alert{唯一}のデータ構造 table
			\uncover<5->{
				\begin{itemize}
					\item 簡単に言うと\alert{連想配列}\only<6->{\textcolor{gray}{$\leftarrow${\Huge{}Perlでは$\cdots\cdots${}?}}}\only<1-5>{、数字もキー}
					\item 関数、数値、文字列、table自身も、オブジェクトは全部詰め込める
					\item \structure{メタテーブル}という機能を用いることにより演算子オーバーロードやプロトタイプベースのオブジェクト指向プログラミンができる
				\end{itemize}}
	\end{itemize}
\end{frame}
\begin{frame}
	\frametitlesubs
	\textcolor{gray}{{\Huge{}Perlでは$\cdots\cdots${}\pause{}ない}}
\end{frame}

\section{Lua VMの調査}
\subsection{Lua VMについて}
\begin{frame}
\frametitlesubs
\begin{itemize}
	\item リオデジャネイロ・カトリカ大学の開発したLua処理系(つまり本家, PUC-Lua)のVM

		\begin{center}
			\textcolor<2->{gray}{ソース$\rightarrow${\color<2->{red}{}バイトコード}$\rightarrow$VMで実行}
		\end{center}
\end{itemize}
\vspace{-1\zw}
\begin{center}
	\only<2->{\alert{の調査と解析}}
	% \alert{\only<2,4>{\only<4>{\hspace{6.5\zw}}の調査}\only<3,4>{\only<3>{\hspace{10.2\zw{}の}}{\only<4>{と}デモ}}}
	\vspace{1\zw}

	\only<3>{\alert{Lua5.2を見る (Lua\LaTeX{}がv5.2ベースなので)}}
\end{center}
\end{frame}
\begin{frame}[fragile]
	\frametitlesubs
	1環境(メインチャンク、関数クロージャ)ごとにレジスタ(slots, constants、upvalues、locals、functions)を用意

	\begin{lstlisting}[language={[5.2]lua},numbers=none]
function hello()
	print("hello")
end\end{lstlisting}

\begin{table}[h]
	\centering
	\begin{tabular}{rcl}
		slots: &2個&(\lstinline|print|と\lstinline|"hello"|をセットするレジスタ)\\
		constants:& 2個&(\lstinline|"print"|、\lstinline|"hello"|)\\
		upvalues:& 1個&(\lstinline|_ENV|\footnote[frame]{グローバル変数は\lstinline|_ENV| tableから持ってくるため})\\
		functions:& 0個&\\
		locals:& 0個&
	\end{tabular}
\end{table}
\end{frame}

\subsection{Bytecode}
\begin{frame}
	\frametitlesubs
	\begin{figure}
		\centering
		\begin{tikzpicture}
			\node[text width=13\zw] (sousa) at (0, 0) {Lua VM 5.3のバイトコード\\を操作したい};
			\node[right = .5 of sousa] (arrow) {$\Rightarrow$};
			\node[right = .5 of arrow, text width=13\zw] {バイトコードのdocumentは\\\alert<2->{ない}};
		\end{tikzpicture}
	\end{figure}

	\pause
	\vspace{3\zw}
	\begin{center}
		$\Rightarrow$ \alert{自分で読み解くしかない}
	\end{center}
\end{frame}
\begin{frame}
	\frametitlesubs

	有志の非公式ドキュメント
	\begin{itemize}
		\item Lua VM 5.3 instructions (bytecodeではない)\footnote{\url{https://github.com/dibyendumajumdar/ravi/blob/master/readthedocs/lua_bytecode_reference.rst}}
		\item Lua VM 5.1 reference\footnote{\url{http://luaforge.net/docman/83/98/ANoFrillsIntroToLua51VMInstructions.pdf}}
	\end{itemize}
	\pause

	Lua VM bytecodeを読むためのツール
	\begin{itemize}
		\item \lstinline{luac -l -l luac.out}
		\item \lstinline{xxd -g 1 luac.out | nvim - -R}\pause
		\item ソースコード\footnote{\url{https://www.lua.org/source}}
	\end{itemize}

	\pause
	\begin{center}
		簡単に言うと\alert{\Huge{}気合}
	\end{center}
\end{frame}
\begin{frame}[fragile]
	\frametitlesubs
	\hspace{-.5\zw}
	\begin{minipage}[t]{.34\textwidth}
		\bgroup\footnotesize
		\begin{lstlisting}[numbers=none,language={[5.3]lua}]
print("hello, world!")
		\end{lstlisting}
		\tiny
		\begin{lstlisting}[numbers=none]
 $ luac -l -l luac.out

main <hello.lua:0,0> (4 instructions at 0x16e79e0)
0+ params, 2 slots, 1 upvalue, 0 locals, 2 constants, 0 functions
  1  [1]  GETTABUP     0 0 -1  ; _ENV "print"
  2  [1]  LOADK        1 -2    ; "hello, world!"
  3  [1]  CALL         0 2 1
  4  [1]  RETURN       0 1
constants (2) for 0x16e79e0:
  1  "print"
  2  "hello, world!"
locals (0) for 0x16e79e0:
upvalues (1) for 0x16e79e0:
  0  _ENV    1       0
		\end{lstlisting}
		\egroup
	\end{minipage}
	\begin{minipage}{.01\textwidth}
		\ 
	\end{minipage}
	\begin{minipage}[t]{.64\textwidth}
		\pause
		\bgroup\tiny
		\begin{lstlisting}[numbers=none]
$ xxd -g 1 luac.out
00000000: 1b 4c 75 61 53 00 19 93 0d 0a 1a 0a 04 08 04 08  .LuaS...........
00000010: 08 78 56 00 00 00 00 00 00 00 00 00 00 00 28 77  .xV...........(w
00000020: 40 01 0b 40 68 65 6c 6c 6f 2e 6c 75 61 00 00 00  @..@hello.lua...
00000030: 00 00 00 00 00 00 02 02 04 00 00 00 06 00 40 00  ..............@.
00000040: 41 40 00 00 24 40 00 01 26 00 80 00 02 00 00 00  A@..$@..&.......
00000050: 04 06 70 72 69 6e 74 04 0e 68 65 6c 6c 6f 2c 20  ..print..hello,
00000060: 77 6f 72 6c 64 21 01 00 00 00 01 00 00 00 00 00  world!..........
00000070: 04 00 00 00 01 00 00 00 01 00 00 00 01 00 00 00  ................
00000080: 01 00 00 00 00 00 00 00 01 00 00 00 05 5f 45 4e  ............._EN
00000090: 56                                                                  V
		\end{lstlisting}
		\egroup\normalfont
		\pause
		\begin{center}
			\vspace{1\zw}
			\LARGE
			\alert{???}
		\end{center}
	\end{minipage}
\end{frame}
\begin{frame}[fragile]
	\frametitlesubs
	\begin{center}
		\begin{minipage}[t]{.4\textwidth}
			\bgroup\tiny
			\begin{lstlisting}[numbers=none]
1b 4c 75 61 53 00 19 93 0d 0a 1a 0a 04 08 04 08
08 78 56 00 00 00 00 00 00 00 00 00 00 00 28 77
40 01 0b 40 68 65 6c 6c 6f 2e 6c 75 61 00 00 00
00 00 00 00 00 00 02 02 04 00 00 00 06 00 40 00
41 40 00 00 24 40 00 01 26 00 80 00 02 00 00 00
04 06 70 72 69 6e 74 04 0e 68 65 6c 6c 6f 2c 20
77 6f 72 6c 64 21 01 00 00 00 01 00 00 00 00 00
04 00 00 00 01 00 00 00 01 00 00 00 01 00 00 00
01 00 00 00 00 00 00 00 01 00 00 00 05 5f 45 4e
56
			\end{lstlisting}
			\egroup
		\end{minipage}
	\end{center}
	\pause
	\begin{figure}[h]
		\def\position{(0.1, -1.76)}
		\tikzset{just here/.style = {above right,inner sep=0mm}}
		\centering
		\begin{tikzpicture}[remember picture]
			\useasboundingbox (0.1,-1.76);
			\coordinate (top) at (-3, 2.3);
			\node[above right = .2 of top] (label) {\textcolor{blue}{header block}};
			\coordinate[right = 6.2 of top] (p1);
			\coordinate[below = 0.55 of p1] (p2);
			\coordinate[left = 5.4 of p2] (p3);
			\coordinate[below = 0.28 of p3] (p4);
			\coordinate[left = 0.8 of p4] (p5);
			\draw[blue, dashed, very thick] (top) -- (p1) -- (p2) -- (p3) -- (p4) -- (p5) -- (top);
		\end{tikzpicture}
	\end{figure}
	\pause
	\begin{figure}[h]
		\tikzset{just here/.style = {above right,inner sep=0mm}}
		\centering
		\begin{tikzpicture}[remember picture]
			\useasboundingbox (0.1,-1.76);
			\coordinate (bottom) at (-3, 1.33);
			\node[below right = .2 of bottom] (label) {\textcolor{red}{function block}};
			\coordinate[right = 6.2 of bottom] (p1);
			\coordinate[above = 2.14 of p1] (p2);
			\coordinate[left = 5.4 of p2] (p3);
			\coordinate[below = .28 of p3] (p4);
			\coordinate[left = .8 of p4] (p5);
			\draw[red, dashed, very thick] (bottom) -- (p1) -- (p2) -- (p3) -- (p4) -- (p5) -- (bottom);
		\end{tikzpicture}
	\end{figure}
\end{frame}



\section{解析}
\section{implementation}

\section{まとめ}
\section{まとめ}
\begin{frame}
    \frametitlesec

    \begin{itemize}
        \item[\coloremoji{🌋}] Algebraic Effectsが楽しい

            ICFP 2018やML Workshop2018にも\\
            AE関連のトピック%
            \footnote{\url{https://icfp18.sigplan.org/event/mlfamilyworkshop-2018-papers-programming-with-abstract-algebraic-effects}}%
            \footnote{\url{https://icfp18.sigplan.org/event/icfp-2018-papers-versatile-event-correlation-with-algebraic-effects}}

        \item[\coloremoji{🉐}] Algebraic Effects使おう

            \begin{itemize}
                \item[\coloremoji{🛠}] インプリいろいろ
                \item[\coloremoji{💪}] なければ自作も可
            \end{itemize}

        \item[\coloremoji{👨‍💻}] 研究やってます
    \end{itemize}
\end{frame}


\skipnexttoc
\section{参考文献}
\begin{frame}
	\frametitlesec
	\begin{itemize}
		\item Lua 5.2 Bytecode and Virtual Machine (Web site)

			{\tiny{\url{http://files.catwell.info/misc/mirror/lua-5.2-bytecode-vm-dirk-laurie/lua52vm.html}}}

		\item A No-Frills Introduction to Lua 5.1 VM Instructions (pdf)

			{\tiny\url{http://luaforge.net/docman/83/98/ANoFrillsIntroToLua51VMInstructions.pdf}}
		\item Source code
			\begin{itemize}
				\item on web

					{\tiny\url{http://www.lua.org/source/5.2/}}
				\item tar

					{\tiny\url{https://www.lua.org/ftp/lua-5.2.4.tar.gz}}

					{\tiny\url{https://www.lua.org/ftp/lua-5.3.2.tar.gz} (stable latest)}
			\end{itemize}
	\end{itemize}
\end{frame}

\skipnexttoc
\section*{}
\begin{frame}
	\begin{center}
		\Huge{}おわり
	\end{center}
\end{frame}
\end{document}
